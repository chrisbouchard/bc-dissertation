\chapter{Notation and Preliminaries}\label{chap:prelims}

We assume the reader is familiar with the usual notations and concepts in term
rewriting systems~\cite{Term} and equational unification~\cite{BaaderSnyd-01}.

\section{Terms and Substitutions}\label{sec:terms-subs}

We consider terms over a ranked signature, usually denoted $\Sigma$, and a
possibly infinite set of variables, usually denoted $\X$. For a function symbol
$f$ in $\Sigma$, we denote its arity by $\Arity(f)$. We denote by
$\Terms(\Sigma, \X)$ the set of all terms over $\Sigma$ and $\X$, and by
$\Terms(\Sigma)$ the set of ground terms, i.e., $\Terms(\Sigma) =
\Terms(\Sigma, \varnothing)$. For a term $t$, we write $\Pos(t)$ for the set of
positions in $t$, and $\FPos(t)$ for the set of nonvariable positions. We write
$\Var(t)$ for the set of variables in $t$, and $\Sig(t)$ for the set of
function symbols in $t$, also called the signature of $t$. The root symbol of
$t$ is denoted $\Root(t)$.

A function $\Subst$ on $\Terms(\Sigma, \X)$ is a \emph{substitution} if it is
an endomorphism over $\Terms(\Sigma, \X)$ and $\Subst(t) = t$ for almost all
terms $t$. We write $\Sub{\Subst}{V}$ for the restriction of $\Subst$ to domain
$V$ such that $(\Sub{\Subst}{V})(x) = x$ for any variable $x$ not in $V$.
Substitutions will often be given as a mapping of variables, e.g., $\{ x
\mapsto a,\, y \mapsto b \}$. A function from variables to terms can be
naturally extended to a substitution. We write $\Sig(\Subst)$ for the signature
of a substitution $\Subst$.

There is a partial order $\LessGeneralThan$ on terms given by $t_1
\LessGeneralThan t_2$ if and only if $t_1 = \Subst(t_2)$ for some substitution
$\Subst$. If $\Subst$ is important, this can be written as $t_1
\LessGeneralThan[\Subst] t_2$. Here $t_1$ is said to be \emph{less general
than}, or \emph{subsumed by}, $t_2$. There is a similar ordering for
substitutions, where $\Subst[1] \LessGeneralThan \Subst[2]$ if and only if
$\Subst[1] = \AltSubst \Compose \Subst[2]$ for some substitution $\AltSubst$.

\section{Equational Theories and Term Rewriting}\label{sec:eq-sys-term-rew}

An \emph{equation} is an ordered pair of terms $(t_1, t_2)$, usually written as
$t_1 \Equals{} t_2$. An \emph{equational theory} (or simply a theory) is a set
of equations, which are called the \emph{axioms} of the theory. An equational
theory $E$ induces an equivalence relation $\Equals{E}$ on terms such that $t_1
\Equals{E} t_2$ if and only if $E \vdash t_1 \Equals{} t_2$. We denote by
$\Sig(E)$ the signature of a theory $E$.

A \emph{rewrite rule} is also an ordered pair of terms $(t_1, t_2)$, usually
written as $t_1 \To{} t_2$. A \emph{term rewriting system} (or simply a rewrite
sytem) is a set of rewrite rules. A rewrite system $R$ induces an equivalence
relation $\Equals{R}$ on terms such that $t_1 \Equals{R} t_2$ if and only if
$t_1 \ToFrom[*]{R} t_2$, where $\ToFrom[*]{R}$ is the reflexive, transitive,
symmetric closure of $\To{R}$. There is also a relation $\JoinTo{R}$, where
$t_1 \JoinTo{R} t_2$ if and only if there is a term $t'$ such that $t_1
\To[*]{R} t' \From[*]{R} t_2$. Then $t_1$ and $t_2$ are said to be
\emph{joinable} in $R$. We denote by $\Sig(R)$ the signature of a rewrite
system $R$.

A rewrite system $R$ is said to be \emph{terminating} if there are no infinite
rewrite sequences, and said to be \emph{confluent} if for all terms $t$, $t_1$,
and $t_2$ such that $t_1 \From[*]{R} t \To[*]{R} t_2$, then $t_1 \JoinTo{R}
t_2$. A rewrite system that is both terminating and confluent is called
\emph{convergent}. If $R$ is a convergent rewrite system, then every term $t$
has a unique \emph{normal form} $\Reduced{t}$ such that $t \To[*]{R}
\Reduced{t}$ and $\Reduced{t}$ is irreducible, and we write $t \To[!]{R}
\Reduced{t}$. For a convergent rewrite system $R$, the relations $\JoinTo{R}$
and $\Equals{R}$ are equivalent.

\section{Unification}\label{sec:unification}

Unification is the problem of solving equations with respect to a particular
equivalence relation. Often, the equivalence relation given by an equational
theory, in which case we refer to the problem as \emph{equational unification}.
Given an theory $E$, an instance of the \emph{unification problem modulo $E$}
(or the \emph{$E$-unification problem}) is a set of equations\footnote{The
symbol $\ueq$ rather than $\Equals{}$ is customary to indicate that these are
equations to be solved rather than axioms.} $\Prob = \{ s_1 \ueq t_1, \dotsc,
s_n \ueq t_n \}$. A solution is a substitution $\Unifier$ such that $\Unifier(s_1)
\Equals{E} \Unifier(t_1) \; \wedge \; \dotsb \; \wedge \; \Unifier(s_n) \Equals{E}
\Unifier(t_n)$. Such a solution is called an \emph{$E$-unifier} of $\Prob$, or
simply a \emph{unifier} if the theory is clear from context. Also, we will often
call $\Prob$ an $E$-unification problem.

The $E$-unification problem comes in three flavors, with decreasing degrees of
restriction on the signature of the input problem $\Prob$. First is
\emph{elementary unification}, in which $\Sig(\Prob) \subseteq \Sig(E)$. Next
is \emph{unification with constants}, where $\Sig(\Prob) \setminus \Sig(E)$
contains only constants. Last is \emph{general unification}, where no
restrictions are placed on $\Sig(\Prob)$.

If $\Unifier$ is an $E$-unifier of a problem $\Prob$, and $\Unifier$ is maximal
among $E$-unifiers of $\Prob$ under the subsumption ordering $\sqsubseteq$, we
say $\Unifier$ is a \emph{most general $E$-unifier} of $\Prob$. If every instance
of the $E$-unification problem has at most one most general unifier, the
$E$-unification problem is said to be \emph{unitary}. If there are at most a
finite number of most general unifiers, it is said to be \emph{finitary}.
Otherwise the problem is said to be \emph{infinitary}.

An \emph{$E$-unification algorithm} is an algorithm that solves the unification
problem modulo $E$. We can think of a unification algorithm $\Alg{}$ as a
relation on unification problems and substitutions such that $\Alg{}(\Prob,
\Unifier)$ if and only if $\Unifier$ is a most general $E$-unifier of $\Prob$.
Equivalently, we can think of $\Alg{}$ as a function from unification problems
to sets of substitutions such that
\[ \Alg{}(\Prob) := \{ \Unifier \mid \Alg{}(\Prob, \Unifier) \} \]
For a given problem $\Prob$, $\Alg{}(\Prob)$ may be an infinite set if
unication modulo $E$ is not finitary. A problem $\Prob$ is $E$-unifiable if and
only if $\Alg{}(\Prob)$ is not the empty set.

\section{Standard Form and DAG-Solved Form}\label{sec:std-form}

A unification problem $\Prob$ is said to be in \emph{standard form} if, for
each equation $t_1 \ueq t_2$ in $\Prob$, $t_1$ is a variable, and $t_2$ is
either a variable, a constant, or a term of the form $f(x_1, \dotsc, x_n)$. A
unification problem $\Prob$ not in standard form can be decomposed to a problem
$\Prob'$ that is in standard form and $|\Prob|$ is at most twice $|\Prob'|$.

In later chapters, we will generally assume that unification problems are given
in standard form.

A problem $\Prob = \{ x_1 \ueq t_1, \dotsc, x_n \ueq t_n \}$ in standard form
is in \emph{dag-solved form} if and only if both of the following conditions
hold:

\begin{enumerate}[(i)]
    \item Each $x_i$ is unique, i.e., $x_i = x_j$ if and only if $i = j$.
    \item If $i < j$, then $x_i$ does not occur in $t_j$.
\end{enumerate}

This second condition gives dag-solved form its name, since the equations can
be thought to form an acyclic graph. For such a problem in dag-solved form, the
substitution
\[\Unifier = \{ x_n \mapsto t_n \} \Compose \dotsb \Compose \{ x_1 \mapsto t_1 \}\]
is always a unifier.

If a problem has a variable that is the left-hand side of two different equations,
e.g., $x \ueq t_1$ and $x \ueq t_2$, then we call this a \emph{peak}. If
$\Root(t_1) = f_1$ and $\Root(t_2) = f_2$, then we call it a $f_1/f_2$-peak.

\section{Dependency Graphs}\label{sec:dep-graphs}

For a unification problem $\Prob$ in standard form, the \emph{dependency graph}
of $\Prob$, written $\DepGraph(\Prob)$, is a labelled directed graph whose
nodes are variables in $\Var(\Prob)$ and whose edges are labelled by pairs from
$\Sig(\Prob) \times \Nat$.  For each equation $x \ueq f(y_1, \dotsc, y_n)$ in
$\Prob$, there is an edge from $x$ to each $y_i$ labelled $(f, i)$.

\section{Many-Sorted Signatures}\label{sec:sorts}

We will be considering many-sorted signatures over a set of sorts $\SortSet$.
Each symbol's arity, rather than being simply a number, is a type signature
composed of sorts from $\SortSet$ --- e.g., $\Arity(f) = s_1 \times \dotsb \times
s_n \to s'$. We assume that $\X$ is partitioned by sort such that $\X =
\biguplus_{s \in \SortSet} \X_{s}$ and variables in $\X_{s}$ can only be
instantiated by terms of sort $s$. We further assume that $\Terms(\Sigma, \X)$
only contains well-typed terms. There is a sort relation such that $t : s$ if
and only if $\Arity(\Root(t))$ has range $s$. Variables are annotated by sorts,
where variable $\V{x}{s} \in \X_s$ can only be instantiated with a term of sort
$s$.

For a signature $\Sigma$, we denote by $\Sorts(\Sigma)$ the set of sorts. We
extend this function to anything defined over a signature, e.g.,
$\Sorts(\EqTheory)$ for the sorts equational theory $\EqTheory$.

As a notational short-hand for the sorts we study in this paper, we will write
a dot over variables of sort $\Elt$. So $\Ve{x}$ and $\V{x}{\Elt}$ are the
same. For variables of sort $\List$, we will write an arrow over the variable.
Thus $\Vl{y}$ and $\V{y}{\List}$ are the same.

\section{Inference Rules}\label{sec:inf-rules}

Our algorithms use inference rules and inference systems to transform
unification problems into solvable forms. The inference rules we use are
not quite the same as inference rules from symbolic logic, so we will briefly
describe their usage.

A \emph{problem pattern} is an expression of the form $S \uplus \{ s_1 \ueq
t_1, \dotsc, s_n \ueq t_n \}$, where $S$ is a set name and $s_i \ueq t_i$ is an
equation. Such a problem pattern is said to match a problem $\Prob$ if and only
if $\Prob = \Prob' \uplus \{ s'_1 \ueq t'_1, \dotsc, s'_n \ueq t'_n \}$, where
$\uplus$ is disjoint union, and each $s'_i = \Subst(s_i)$ and $t'_i =
\Subst(t_i)$ for some substitution $\Subst$.

Our notion of an inference rule is different from the standard notion from
symbolic logic. For our purposes, an \emph{inference rule} is a tripple
containing the following:
\begin{enumerate}[(i)]
    \item A problem pattern, called the \emph{premise}
    \item A problem, called the \emph{conclusion}
    \item An optional condition
\end{enumerate}
and is written:
\[\CondInfDef
    {S}{s_1 \ueq t_1, \; \dotsc, \; s_m \ueq t_m}
    {S}{s'_1 \ueq t'_1, \; \dotsc, \; s'_n \ueq t'_n}
    {if \textit{condition}}
\]
An inference rule $\rho$ can be applied to a problem $\Prob$ if its condition
holds (or was ommitted) and its premise matches $\Prob$ with substitution
$\Subst$. We then obtain a new problem $\Prob'$ and write $\Prob \InfTo{\rho}
\Prob'$, where
\[ \Prob' := S \cup \{ \Subst(s'_1) \ueq \Subst(t'_1), \dotsc, \Subst(s'_n)
\ueq \Subst(t'_n) \} \]
In this case, $\Prob$ may be referred to as the premise, and $\Prob'$ as the
conclusion, even though technically they are \emph{instances of} the premise
and conclusion.

A \emph{failure inference rule} is an inference rule with the special
conclusion $\Fail$. These inference rules are applied as normal, but the result
indicates that the original problem was unsolvable.

An \emph{inference system} is a set of inference rules. An inference system
$\INF{}$ can be applied to a problem $\Prob$ to obtain a new problem $\Prob'$,
written $\Prob \InfTo{\INF{}} \Prob'$, if and only if there is an inference rule
$\rho$ in $\INF{}$ such that $\Prob \InfTo{\rho} \Prob'$. If $\Prob$ and
$\Prob'$ are problems such that $\Prob \InfTo[*]{\INF{}} \Prob'$, and no rule
in $\INF{}$ is applicable to $\Prob'$, we write $\Prob \InfTo[!]{\INF{}}
\Prob'$.

Note that the process for choosing inference rules is left unspecified. Often
more than once inference rule in a system will be applicable to the same
problem, and there is no guarantee that different paths will lead to the same
result (i.e., inference systems are not assumed to be confluent) or that a path
will terminate at all. An inference system should be accompanied by a
\emph{strategy} --- an algorithm for chosing an inference rule given a problem.

An inference rule $\rho$ is said to be \emph{sound} if and only if, for every
pair of problems $\Prob$ and $\Prob'$ such that $\Prob \InfTo{\rho} \Prob'$,
every unifier of $\Prob'$ is a unifier of $\Prob$. Such a rule is said to be
\emph{deterministic} if and only if every unifier of $\Prob$ is a unifier of
$\Prob'$.

