\chapter{Notation and Preliminaries}\label{chap:prelims}

We assume the reader is familiar with the usual notations and concepts in term
rewriting systems~\cite{Term} and equational unification~\cite{BaaderSnyd-01}.

\section{Terms and Substitutions}\label{sec:terms-subs}

We consider terms over a ranked signature, usually denoted $\Sigma$, and a
possibly infinite set of variables, usually denoted $\X$. For a function symbol
$f$ in $\Sigma$, we denote its arity by $\Arity(f)$. We denote by
$\Terms(\Sigma, \X)$ the set of all terms over $\Sigma$ and $\X$, and by
$\Terms(\Sigma)$ the set of ground terms, i.e., $\Terms(\Sigma) =
\Terms(\Sigma, \varnothing)$. For a term $t$, we write $\Pos(t)$ for the set of
positions in $t$, and $\FPos(t)$ for the set of nonvariable positions. We write
$\Var(t)$ for the set of variables in $t$, and $\Sig(t)$ for the set of
function symbols in $t$, also called the signature of $t$. The root symbol of
$t$ is denoted $\Root(t)$.

A function $\sigma$ on $\Terms(\Sigma, \X)$ is a \emph{substitution} if it is
an endomorphism over $\Terms(\Sigma, \X)$ and $\sigma(t) = t$ for almost all
terms $t$. We write $\Sub{\sigma}{V}$ for the restriction of $\sigma$ to domain
$V$ such that $(\Sub{\sigma}{V})(x) = x$ for any variable $x$ not in $V$.
Substitutions will often be given as a mapping of variables, e.g., $\{ x
\mapsto a,\, y \mapsto b \}$. A function from variables to terms can be
naturally extended to a substitution. We write $\Sig(\sigma)$ for the signature
of a substitution $\sigma$.

There is a partial order $\LessGeneralThan$ on terms given by $t_1
\LessGeneralThan t_2$ if and only if $t_1 = \sigma(t_2)$ for some substitution
$\sigma$. If $\sigma$ is important, this can be written as $t_1
\LessGeneralThan[\sigma] t_2$. Here $t_1$ is said to be \emph{less general
than}, or \emph{subsumed by}, $t_2$. There is a similar ordering for
substitutions, where $\sigma_1 \LessGeneralThan \sigma_2$ if and only if
$\sigma_1 = \theta \compcirc \sigma_2$ for some substitution $\theta$.

\section{Equational Theories and Term Rewriting}\label{sec:eq-sys-term-rew}

An \emph{equation} is an ordered pair of terms $(t_1, t_2)$, usually written as
$t_1 \Equals{} t_2$. An \emph{equational theory} (or simply a theory) is a set
of equations, which are called the \emph{axioms} of the theory. An equational
theory $E$ induces an equivalence relation $\Equals{E}$ on terms such that $t_1
\Equals{E} t_2$ if and only if $E \vdash t_1 \Equals{} t_2$. We denote by
$\Sig(E)$ the signature of a theory $E$.

A \emph{rewrite rule} is also an ordered pair of terms $(t_1, t_2)$, usually
written as $t_1 \To{} t_2$. A \emph{term rewriting system} (or simply a rewrite
sytem) is a set of rewrite rules. A rewrite system $R$ induces an equivalence
relation $\Equals{R}$ on terms such that $t_1 \Equals{R} t_2$ if and only if
$t_1 \ToFrom[*]{R} t_2$, where $\ToFrom[*]{R}$ is the reflexive, transitive,
symmetric closure of $\To{R}$. There is also a relation $\JoinTo{R}$, where
$t_1 \JoinTo{R} t_2$ if and only if there is a term $t'$ such that $t_1
\To[*]{R} t' \From[*]{R} t_2$. Then $t_1$ and $t_2$ are said to be
\emph{joinable} in $R$. We denote by $\Sig(R)$ the signature of a rewrite
system $R$.

A rewrite system $R$ is said to be \emph{terminating} if there are no infinite
rewrite sequences, and said to be \emph{confluent} if for all terms $t$, $t_1$,
and $t_2$ such that $t_1 \From[*]{R} t \To[*]{R} t_2$, then $t_1 \JoinTo{R}
t_2$. A rewrite system that is both terminating and confluent is called
\emph{convergent}. If $R$ is a convergent rewrite system, then every term $t$
has a unique \emph{normal form} $\Reduced{t}$ such that $t \To[*]{R}
\Reduced{t}$ and $\Reduced{t}$ is irreducible, and we write $t \To[!]{R}
\Reduced{t}$. For a convergent rewrite system $R$, the relations $\JoinTo{R}$
and $\Equals{R}$ are equivalent.

\section{Unification}\label{sec:unification}

Unification is the problem of solving equations with respect to a particular
equivalence relation. Often, the equivalence relation given by an equational
theory, in which case we refer to the problem as \emph{equational unification}.
Given an theory $E$, an instance of the \emph{unification problem modulo $E$}
(or the \emph{$E$-unification problem}) is a set of equations\footnote{The
symbol $\ueq$ rather than $\Equals{}$ is customary to indicate that these are
equations to be solved rather than axioms.} $\Prob = \{ s_1 \ueq t_1, \dotsc,
s_n \ueq t_n \}$. A solution is a substitution $\sigma$ such that $\sigma(s_1)
\Equals{E} \sigma(t_1) \; \wedge \; \dotsb \; \wedge \; \sigma(s_n) \Equals{E}
\sigma(t_n)$. Such a solution is called an \emph{$E$-unifier} of $\Prob$, or
simply a \emph{unifier} if the theory is clear from context.

The $E$-unification problem comes in three flavors, with decreasing degrees of
restriction on the signature of the input problem $\Prob$. First is
\emph{elementary unification}, in which $\Sig(\Prob) \subseteq \Sig(E)$. Next
is \emph{unification with constants}, where $\Sig(\Prob) \setminus \Sig(E)$
contains only constants. Last is \emph{general unification}, where no
restrictions are placed on $\Sig(\Prob)$.

If every instance of the $E$-unification problem has at most one solution, the
$E$-unification problem is said to be \emph{unitary}. If there are at most a
finite number of solutions, it is said to be \emph{finitary}. Otherwise the
problem is said to be \emph{infinitary}.

\section{Standard Form and DAG-Solved Form}\label{sec:std-form}

A unification problem $\Prob$ is said to be in \emph{standard form} if, for
each equation $t_1 \ueq t_2$ in $\Prob$, $t_1$ is a variable, and $t_2$ is
either a variable, a constant, or a term of the form $f(x_1, \dotsc, x_n)$. A
unification problem $\Prob$ not in standard form can be decomposed to a problem
$\Prob'$ that is in standard form and $|\Prob|$ is at most twice $|\Prob'|$.

A problem $\Prob = \{ x_1 \ueq t_1, \dotsc, x_n \ueq t_n \}$ in standard form
is in \emph{dag-solved form} if and only if both of the following conditions
hold:

\begin{enumerate}[(i)]
    \item Each $x_i$ is unique, i.e., $x_i = x_j$ if and only if $i = j$.
    \item If $i < j$, then $x_i$ does not occur in $t_j$.
\end{enumerate}

This second condition gives dag-solved form its name, since the equations can
be thought to form an acyclic graph. For such a problem in
dag-solved form, the substitution
\[\sigma = \{ x_n \mapsto t_n \} \compcirc \dotsb \compcirc \{ x_1 \mapsto t_1 \}\]
is always a solution.

\section{Dependency Graphs}\label{sec:dep-graphs}

For a unification problem $\Prob$ in standard form, the \emph{dependency graph}
of $\Prob$, written $\DepGraph(\Prob)$, is a labelled directed graph whose
nodes are variables in $\Var(\Prob)$ and whose edges are labelled by pairs from
$\Sig(\Prob) \times \Nat$.  For each equation $x \ueq f(y_1, \dotsc, y_n)$ in
$\Prob$, there is an edge from $x$ to each $y_i$ labelled $(f, i)$.

\section{Many-Sorted Signatures}\label{sec:sorts}

We will be considering many-sorted signatures over a set of sorts $\Sorts$.
Each symbol's arity, rather than being simply a number, is a type signature
composed of sorts from $\Sorts$ --- e.g., $\Arity(f) = s_1 \times \dotsb \times
s_n \to s'$. We assume that $\X$ is partitioned by sort such that $\X =
\biguplus_{s \in \Sorts} \X_{s}$ and variables in $\X_{s}$ can only be
instantiated by terms of sort $s$. We further assume that $\Terms(\Sigma, \X)$
only contains well-typed terms. There is a sort relation such that $t : s$ if
and only if $\Arity(\Root(t))$ has range $s$. Variables are annotated by sorts,
where variable $\V{x}{s} \in \X_s$ can only be instantiated with a term of sort
$s$.

As a notational short-hand for the sorts we study in this paper, any variable
without a sort annotation is assumed to have sort $\Elt$. So $\Ve{x}$ and
$\V{x}{\Elt}$ are the same. For variables of sort $\List$, we will place a bar
over the variable. Thus $\Vl{y}$ and $\V{y}{\List}$ are the same.

\section{Inference Rules}\label{sec:inf-rules}

