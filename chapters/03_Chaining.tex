\chapter{Chaining}\label{chap:chaining}

The \Oper{chain} operation, also known as \Oper{scan} in some circles, is a
list operation where a function is applied successively to elements of a list
and the results are stored in a new list.
\[ \text{\Oper{chain}} \colon (a \times b \to a) \to a \to b \; \List \to a \; \List \]
The operation is similar to the
\Oper{right-fold} or \Oper{accumulate} operations, but instead of returning
only the result value, a list of the partial results is returned. As a function
in ML, \Oper{chain} could be written
\begin{lstlisting}[language=ML, style=Inline]
fun chain f z []      = []
  | chain f z (x::ls) = f (z,x) :: chain f (f (z,x)) ls;
\end{lstlisting}
The \Oper{chain} function is a higher-level function, taking a function
$f$ as its first argument. The second argument $z$ is the initial value or
\emph{initialization vector}. The last argument is the list. The function
$f$ is applied to each element in the list and the current $z$, and the
result is stored in the list and passed as the new initialization vector
for the tail.

\begin{Example}
We can combine \Oper{chain} with addition, in
    which case we get partial sums.
    \begin{lstlisting}[language=ML, style=Inline]
- chain (op +) 0 [1, 2, 3, 4];
> val it = [1, 3, 6, 10] : int list
    \end{lstlisting}
\end{Example}

Since we are dealing with first order theorems, we will usually consider
\Oper{chain} as an un-curried function with a fixed function $f$.

\section{Cipher Block Chaining}\label{sec:cbc}

The technique of \emph{cipher block chaining} was invented in 1978 at
IBM~\cite{ehrsam1978message} and is currently used in the SSL and TLS
cryptographic protocols~\cite{rfc5246}. It is a mode of operation in which one
encrypts blocks of a message by masking the current block with the previous
encrypted block, then applying the block-level encryption function to the
result. This forces an attacker to decrypt all previous blocks before
decrypting a later block, preventing him or her from attacking individual
blocks in parallel.

Formally, if a message's plaintext contains blocks $[M_1, \dotsc, M_n]$, then
the ciphertext will contain blocks $[C_1, \dotsc, C_n]$ such that:
\begin{align*}
    C_1 &= e_k(M_1 \oplus IV) &
    C_{i+1} &= e_k(M_i \oplus C_i)
\end{align*}
where $e_k$ is a function for block-level encryption with a key $k$, $\oplus$
is XOR, and $IV$ is an initialization vector.

The block chaining operation can be seen as an instance of \Oper{chain}, where
the function is XOR. In later sections, we will use block chaining as a
motivating example of studying chaining in general.

\section{\texorpdfstring{$\BC$}{BC} Theory}\label{sec:bc-theory}

\section{Element Theory \texorpdfstring{$\HH$}{H}}\label{sec:elt-theory}

\section{Inductive Theory}\label{sec:inductive-theory}

\section{Cancellativity and Semicancellativity}\label{sec:cancel}

