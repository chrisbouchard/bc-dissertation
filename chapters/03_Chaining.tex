\chapter{Chaining}\label{chap:chaining}

The \Oper{chain} operation, also known as \Oper{scan} in some circles, is a
list operation where a function is applied successively to elements of a list
and the results are stored in a new list.
\[ \text{\Oper{chain}} \colon (a \times b \to a) \to a \to b \; \List \to a \; \List \]
The operation is similar to the
\Oper{right-fold} or \Oper{accumulate} operations, but instead of returning
only the result value, a list of the partial results is returned. As a function
in ML, \Oper{chain} could be written
\begin{lstlisting}[language=ML, style=Inline]
fun chain f z []      = []
  | chain f z (x::ls) = f (z,x) :: chain f (f (z,x)) ls;
\end{lstlisting}
The \Oper{chain} function is a higher-level function, taking a function
$f$ as its first argument. The second argument $z$ is the initial value or
\emph{initialization vector}. The last argument is the list. The function
$f$ is applied to each element in the list and the current $z$, and the
result is stored in the list and passed as the new initialization vector
for the tail.

\begin{Example}
We can combine \Oper{chain} with addition, in
    which case we get partial sums.
    \begin{lstlisting}[language=ML, style=Inline]
- chain (op +) 0 [1, 2, 3, 4];
> val it = [1, 3, 6, 10] : int list
    \end{lstlisting}
\end{Example}

Since we are dealing with first order theorems, we will usually consider
\Oper{chain} as an un-curried function with a fixed function $f$.



\section{Cipher Block Chaining}\label{sec:cbc}

The technique of \emph{cipher block chaining} was invented in 1978 at
IBM~\cite{ehrsam1978message} and is currently used in the SSL and TLS
cryptographic protocols~\cite{rfc5246}. It is a mode of operation in which one
encrypts blocks of a message by masking the current block with the previous
encrypted block, then applying the block-level encryption function to the
result. This forces an attacker to decrypt all previous blocks before
decrypting a later block, preventing him or her from attacking individual
blocks in parallel.

Formally, if a message's plaintext contains blocks $[M_1, \dotsc, M_n]$, then
the ciphertext will contain blocks $[C_1, \dotsc, C_n]$ such that:
\begin{align*}
    C_1 &= e_k(M_1 \oplus IV) &
    C_{i+1} &= e_k(M_i \oplus C_i)
\end{align*}
where $e_k$ is a function for block-level encryption with a key $k$, $\oplus$
is bitwise XOR, and $IV$ is an initialization vector.

The block chaining operation can be seen as an instance of \Oper{chain}, where
the function is XOR\@. In later sections, we will use block chaining as a
motivating example of studying chaining in general.



\section{List Terms}
To study chaining, we must first represent lists as terms. We use two sorts,
$\Elt$ and $\List$, and the standard list LISP-style constructors, $\Cons\colon
\Elt \times \List \to \List$ and $\Nil\colon \List$. A term of sort $\Elt$ is
said to be an \emph{element term}, and similarly a term of sort $\List$ is a
\emph{list term}. A list $[a, b, c]$ would be represented by the list term
$\Cons(a, \Cons(b, \Cons(c, \Nil)))$, where $a$, $b$, and $c$ are element
constants.

We disallow list constants other than $\Nil$, so the only ground list terms are
``proper'' $\Nil$-terminated lists. A list can also be terminated by a
variable, e.g. $[a, b \mid \Vl{x}]$ (in Prolog notation), which is represented
by the term $\Cons(a, \Cons(b, \Vl{x}))$.

\begin{Definition}
    Let $\Sigma$ be a signature containing $\Nil$, and let $E$ be an equational
    theory defined over $\Sigma$. We say a term $t$ is \emph{nonnil} if and
    only if $t \not\Equals{\EqTheory} \Nil$.

    As an extension, let $\Prob$ be an $\EqTheory$-unification problem. We say
    a list variable $\Vl{x}$ is nonnil if and only if $\Unifier(\Vl{x})$ is
    nonnil for every $\EqTheory$-unifier $\Unifier$ of $\Prob$.
\end{Definition}

\begin{Definition}
    Let $\Sigma$ be a signature containing $\Cons$, and let $\X$ be a set of
    variables. We define a function $\Length \colon \Terms(\Sigma, \X) \to
    \Nat$ such that $\Length(t)$ is the number of occurrences of the symbol
    $\Cons$ in term $t$.
\end{Definition}

It is not difficult to see that this definition of $\Length$ matches with the
standard concept of the length of a list, assuming that all other list function
symbols are length-preserving operations on lists.



\section{Chaining Theory $\BC$}\label{sec:bc-theory}
To study the chaining operation, we will create an equational theory $\BC$ such
that chaining is a model. Our theory will have two sorts, $\Elt$ and $\List$,
and the following signature:
\begin{align*}
    \Bc &\colon \Elt \times \List \to \List &
    \Cons &\colon \Elt \times \List \to \List \\
    \Ff &\colon \Elt \times \Elt \to \Elt &
    \Nil &\colon \List
\end{align*}

Here $\Bc$ is our chaining operator (\underline{b}lock \underline{c}hain),
$\Ff$ is the fixed function to be chained, and $\Cons$ and $\Nil$ are the
standard list constructors. We define $\BC$ to have the following two axioms:
\begin{align*}
    \Bc(\Ve{x}, \Nil) &\Equals{} \Nil \\
    \Bc(\Ve{x}, \Cons(\Ve{y}, \Vl{z}))
    &\Equals{} \Cons(\Ff(\Ve{y}, \Ve{x}), \Bc(\Ff(\Ve{y}, \Ve{x}), \Vl{z}))
\end{align*}
We can orient these rules from left to right to create a rewrite system $\RBC$:
\begin{align*}
    \Bc(\Ve{x}, \Nil) &\To{} \Nil \\
    \Bc(\Ve{x}, \Cons(\Ve{y}, \Vl{z}))
    &\To{} \Cons(\Ff(\Ve{y}, \Ve{x}), \Bc(\Ff(\Ve{y}, \Ve{x}), \Vl{z}))
\end{align*}

To simplify our notation, we will write $\ToBC$ rather than $\To{\RBC}$.

\begin{Lemma}
    The rewrite system $\RBC$ is convergent.
\end{Lemma}
\begin{proof}
    It is straightforward that $\RBC$ is confluent, because there are no
    critical pairs ($\Cons$ and $\Nil$ clash). To prove $\RBC$ terminating,
    note that all terms have finite size and depth. The first rule removes an
    occurrence of $\Bc$ with each application and must terminate. The second
    preserves the size of the term, but increases the depth of an occurrence of
    $\Bc$. Eventually all occurrences of $\Bc$ will be lower than the
    occurrences of $\Cons$, so the rule must terminate. So $\RBC$ is confluent
    and terminating and thus it is convergent.
\end{proof}

The chaining operation preserves the lengths of lists. We can thus show that
our $\Length$ function still makes sense.

\begin{Lemma}\label{lemma:rbc-preserve-cons}
    Let $t_1$ and $t_2$ be terms such that $t_1 \ToBC t_2$. Then $\Length(t_1)
    = \Length(t_2)$.
\end{Lemma}
\begin{proof}
    Both rules of $\RBC$ have the same number of occurrences of $\Cons$ on each
    side. Note that element variables cannot be assigned a term containing
    occurrences of $\Cons$, because the only symbol with $\Elt$ as its codomain
    is $\Ff$, which does not have $\List$ in its domain. Further, in the second
    rule, both sides have the same number of occurrences of $\Cons$, and both
    sides have the same number of occurrences of $\Vl{z}$. Thus both rules
    preserve the number of occurrences of $\Cons$.
\end{proof}

\begin{Corollary}\label{cor:bc-preserve-cons}
    Let $t_1$ and $t_2$ be terms such that $t_1 \EqualsBC t_2$. Then
    $\Length(t_1) = \Length(t_2)$.
\end{Corollary}
\begin{proof}
    Since $\RBC$ is convergent, $t_1 \EqualsBC t_2$ if and only if $t_1
    \JoinToBC t_2$. Thus the corollary follows from the fact that each step of
    the rewrite proof preserves the lengths of the terms.
\end{proof}



\section{Element Theory \texorpdfstring{$\FF$}{F}}\label{sec:elt-theory}

Since the original chaining operation is a higher-order function, we would like
some way to make $\Ff$ behave like different functions. We do this using an
equational theory $\FF$ called the \emph{element theory}. This theory defines
the function symbol $\Ff\colon \Elt \times \Elt \to \Elt$. A theory $\FF$ must
satisfy four conditions to be an element theory:
\begin{enumerate}[(i)]
    % \item $\FF$ and $\BC$ are \emph{disjoint except for $\Ff$}, i.e.,
    %     $\Sig(\FF) \, \cap \, \Sig(\BC) \subseteq \{ \Ff \}$.
    \item $\List \not\in \Sorts(\FF)$.
    \item $\FF$ is \emph{subterm-collapse-free for $\Ff$}, i.e., there is no
        term $t$ such that $\Root(t) = \Ff$ and $t \EqualsF \Sub{t}{p}$ for $p
        \neq \Empty$.
    \item $\Ff$ is \emph{semicancellative in $\FF$}, i.e., if $\Ff(t_1, t_2)
        \EqualsF \Ff(t_3, t_4)$, then $t_1 \EqualsF t_3 \, \Iff \, t_2 \EqualsF
        t_4$.
    \item The $\FF$-unification problem is finitary.
\end{enumerate}

\begin{Lemma}
    The empty theory $\varnothing$ is an element theory.
\end{Lemma}
\begin{proof}
    The first condition is vacuously true, since the empty theory has no
    signature. For the remaining conditions, recall that $t_1
    \Equals{\varnothing} t_2$ if and only if $t_1 = t_2$. Thus the second
    condition holds because a term cannot be identically equal to a proper
    subterm of itself. The third condition holds because, if $\Ff(t_1, t_2)
    \Equals{\varnothing} \Ff(t_3, t_4)$, then by decomposition $t_1
    \Equals{\varnothing} t_3$ and $t_2 \Equals{\varnothing} t_4$. Finally, the
    fourth condition holds because unification modulo the empty theory is
    unitary.
\end{proof}

% Let $E_\oplus$ be the following equational theory:
% \begin{align*}
%     x \oplus x &\Equals{} 0 \\
%     x \oplus 0 &\Equals{} x \\
%     x \oplus y &\Equals{} y \oplus x \\
%     (x \oplus y) \oplus z &\Equals{} x \oplus (y \oplus z)
%     \intertext{One model for $E_\oplus$ is bits under bitwise XOR. Now let
%     $\FF_\oplus$ be $E_\oplus$ with the following additional axiom:}
%     \Ff(x, y) &\Equals{} e(x \oplus y)
% \end{align*}
% 
% \begin{Lemma}
%     The theory $\FF_\oplus$ is an element theory.
% \end{Lemma}
% \begin{proof}
%     The first condition clearly holds, since $\Sig(\FF_\oplus) = \{ e, \Ff,
%     \oplus, 0 \}$. For the second condition, note that every $\Ff$-rooted
%     term is $\FF_\oplus$-equivalent to an $e$-rooted term, and $e$ is unintereted.
% 
%     To prove the third condition, consider terms $t_1$, $t_2$, $t_3$ and $t_4$
%     such that
%     \[\Ff(t_1, t_2) \Equals{\FF_\oplus} \Ff(t_3, t_4)\]
%     % Suppose $\Ff$ is not semi-cancellative. Without loss
%     % of generality, assume $t_1 \Equals{\FF_\oplus} t_3$ but $t_2 \not\Equals{\FF_\oplus} t_4$.
%     Since $\Ff(t_1, t_2) \Equals{\FF_\oplus} e(t_1 \oplus t_2)$ and $\Ff(t_3,
%     t_4) \Equals{\FF_\oplus} e(t_3 \oplus t_4)$, and since $e$ is an uninterpreted
%     function symbol, we have that
%     \[t_1 \oplus t_2 \Equals{\FF_\oplus} t_3 \oplus t_4\]
%     There are then three possibilities for the terms:
%     \begin{enumerate}[(i)]
%         \item $t_1 \Equals{\FF_\oplus} t_3$ and $t_2 \Equals{\FF_\oplus} t_4$
%         \item $t_1 \Equals{\FF_\oplus} t_2$ and $t_3 \Equals{\FF_\oplus} t_4$
%         \item $t_1 \Equals{\FF_\oplus} t_4$ and $t_2 \Equals{\FF_\oplus} t_3$
%     \end{enumerate}
%     Suppose, without loss of generality, that $t_1 \Equals{\FF_\oplus} t_3$. In the
%     first case, we have immediately that $t_2 \Equals{\FF_\oplus} t_4$. In the
%     second and third cases, by transitivity of $\Equals{\FF_\oplus}$, we find that
%     \[t_1 \Equals{\FF_\oplus} t_2 \Equals{\FF_\oplus} t_3 \Equals{\FF_\oplus} t_4 \]
%     and thus $t_2 \Equals{\FF_\oplus} t_4$.
% \end{proof}



\section{Combined Theory $(\BCUF)$}\label{sec:cancel}

Suppose we have an element theory $\FF$. By definition $\Ff$ is
semicancellative in $\FF$, and thus $\Ff$ is semicancellative in $\BCUF$. We
would like to extend this property to $\Bc$ and say that $\Bc$ is
semicancellative in $\BCUF$, but unfortunately this is not so. Consider the
equivalence
\[\Bc(a, \Nil) \EqualsBCF \Bc(b, \Nil)\]
where $a$ and $b$ are distinct constants. Of course $\Nil \EqualsBCF \Nil$, but
$a \not\EqualsBCF b$. Thus $\Bc$ is not semicancellative. We can, however, say
the following.

\begin{Lemma}\label{lemma:bc-listless-cancel}
    Let $t_1$, $t_2$, $t_3$, and $t_4$ be terms containing no occurrences of
    $\Cons$ or $\Nil$ such that $\Bc(t_1, t_2) \EqualsBCF \Bc(t_3, t_4)$. Then
    $t_1 \EqualsBCF t_3$ and $t_2 \EqualsBCF t_4$.
\end{Lemma}

\begin{proof}
    If there are no occurrences of $\Cons$ or $\Nil$, then the $\BC$ axioms are
    inapplicable. So $\Bc(t_1, t_2) \EqualsF \Bc(t_3, t_4)$, and the symbol
    $\Bc$ is essentially a constructor. Thus $\Bc$ is cancellative, and the
    lemma follows.
\end{proof}

So, if we disallow the list constructors, $\Bc$ is cancellative. This is
relevant for equivalences such as $\Bc(\Ve{u}, \Vl{v}) \EqualsBCF \Bc(\Ve{w},
\Vl{x})$, where we can deduce that $\Ve{u} \EqualsBCF \Ve{w}$ and $\Vl{v}
\EqualsBCF \Vl{x}$. If we allow arbitrary terms, we can still get a partial
result.

\begin{Lemma}\label{lemma:bc-left-cancel}
    Let $\FF$ be an element theory, and let $t_1$, $t_2$, $t_3$, and $t_4$ be
    terms such that
    \begin{equation}\label{eq:bc-left-cancel}
        \Bc(t_1, t_2) \EqualsBCF \Bc(t_3, t_4). \tag{$\ast$}
    \end{equation}
    If $t_1 \EqualsBCF t_3$, then $t_2 \EqualsBCF t_4$.
\end{Lemma}

\begin{proof}
    We will prove the lemma using structural induction on the lengths of $t_2$
    and $t_4$. We assume that the lemma holds for any ground terms
    $t'_1$, $t'_2$, $t'_3$, and $t'_4$ such that
    \[\Bc(t'_1, t'_2) \EqualsBCF \Bc(t'_3, t'_4)\]
    and
    \[\Length(t'_2) + \Length(t'_4) < \Length(t_2) + \Length(t_4).\]
    Note that $t_1$ and $t_3$ are element terms and thus cannot contain
    occurrences of $\Cons$.

    Suppose $t_1 \EqualsBCF t_3$. We first consider the base case when
    $\Length(t_2) + \Length(t_4) = 0$. If $t_2 \EqualsBCF t_4 \EqualsBCF \Nil$
    then we are done. Otherwise, $t_2$ and $t_4$ do not contain $\Nil$, and
    the result follows from \cref{lemma:bc-listless-cancel}.

    In the inductive case, neither $t_2$ nor $t_4$ can be equivalent to $\Nil$,
    because we showed in \cref{cor:bc-preserve-cons} that $\BC$ preserves the
    number of $\Cons$ symbols. We see then that
    \begin{align*}
        t_2 &\EqualsBCF \Cons(t_{21}, t_{22}) \\
        t_4 &\EqualsBCF \Cons(t_{41}, t_{42})
    \end{align*}
    for some ground terms $t_{21}$, $t_{22}$, $t_{41}$, and $t_{42}$ in
    $\RBC$-normal form. We can rewrite the left- and right-hand sides of
    \cref{eq:bc-left-cancel} to obtain the following:
    \[\Bc(t_1, \Cons(t_{21}, t_{22})) \EqualsBCF \Bc(t_3, \Cons(t_{41}, t_{42})).\]
    At this point, we can apply the first $\BC$ axiom and arrive at
    \[\Cons(\Ff(t_{21}, t_1), \Bc(\Ff(t_{21}, t_1), t_{22})) \EqualsBCF
    \Cons(\Ff(t_{41}, t_3), \Bc(\Ff(t_{41}, t_3), t_{42}))\]
    which, since $\Cons$ is cancellative, decomposes into
    \begin{align*}
        \Ff(t_{21}, t_1) &\EqualsBCF \Ff(t_{41}, t_3) \\
        \Bc(\Ff(t_{21}, t_1), t_{22}) &\EqualsBCF \Bc(\Ff(t_{41}, t_3), t_{42}).
    \end{align*}
    Recall that $f$ is semicancellative and that, by assumption, $t_1 \EqualsBCF
    t_3$. Thus, we have that $t_{21} \EqualsBCF t_{41}$. Note that $t_{22}$ and
    $t_{42}$ each have one fewer occurrence of $\Cons$ than $t_2$ and $t_4$,
    respectively, so
    \[\Length(t_{22}) + \Length(t_{42}) < \Length(t_2) + \Length(t_4)\]
    and we can apply our inductive hypothesis to see that $t_{22} \EqualsBCF
    t_{42}$. So, we finally see that
    \[t_2 \EqualsBCF \Cons(t_{21}, t_{22}) \EqualsBCF \Cons(t_{41}, t_{42}) \EqualsBCF t_4\]
    and the lemma is proven.
\end{proof}

Essentially, $\Bc$ is semicancellative in one direction. We call this property
\emph{left-cancel\-lativity}. We saw in the example above that $\Nil$ causes
trouble. However, if we restrict our attention to nonempty lists, we can say
something stronger.

\begin{Theorem}[Conditional semicancellativity of $\Bc$]\label{thm:bc-semi-cancel}
    Let $\FF$ be an element theory, and let $t_1$, $t_2$, $t_3$, and $t_4$ be
    terms such that $t_2$ and $t_4$ are nonnil and
    \addtocounter{equation}{1}
    \begin{equation}\label{eq:bc-semi-cancel}
        \Bc(t_1, t_2) \EqualsBCF \Bc(t_3, t_4). \tag{$\ast$}
    \end{equation}
    Then $t_1 \EqualsBCF t_3$ if and only if $t_2 \EqualsBCF t_4$.
\end{Theorem}

\begin{proof}
    For the `only if' direction, the lemma follows from
    \cref{lemma:bc-left-cancel}.

    For the `if' direction, since $t_2$ and $t_4$ are not equivalent to $\Nil$,
    either $t_2$ and $t_4$ do not contain $\Cons$, in which case the result
    follows from \cref{lemma:bc-listless-cancel}, or
    \begin{align*}
        t_2 &\EqualsBCF \Cons(t_{21}, t_{22}) \\
        t_4 &\EqualsBCF \Cons(t_{41}, t_{42})
    \end{align*}
    for some terms $t_{21}$, $t_{22}$, $t_{41}$, and $t_{42}$ in $\RBC$-normal
    form. We can rewrite the left- and right-hand sides of
    \cref{eq:bc-semi-cancel} to obtain the following:
    \[\Bc(t_1, \Cons(t_{21}, t_{22})) \EqualsBCF \Bc(t_3, \Cons(t_{41}, t_{42})).\]
    At this point, we can apply the first $\BC$ axiom and arrive at
    \[\Cons(\Ff(t_{21}, t_1), \Bc(\Ff(t_{21}, t_1), t_{22})) \EqualsBCF
    \Cons(\Ff(t_{41}, t_3), \Bc(\Ff(t_{41}, t_3), t_{42}))\]
    which, since $\Cons$ is cancellative, decomposes into
    \begin{align*}
        \Ff(t_{21}, t_1) &\EqualsBCF \Ff(t_{41}, t_3) \\
        \Bc(\Ff(t_{21}, t_1), t_{22}) &\EqualsBCF \Bc(\Ff(t_{41}, t_3), t_{42}).
    \end{align*}

    Suppose $t_2 \EqualsBCF t_4$. Since $\Cons$ is cancellative, we then have
    that $t_{21} \EqualsBCF t_{41}$ and $t_{22} \EqualsBCF t_{42}$. Recall that
    $f$ is semicancellative, so $t_1 \EqualsBCF t_3$ and the theorem is proven.
\end{proof}

\begin{Corollary}\label{cor:bc-eq-tail}
    Let $\FF$ be an element theory, and let $t_1$, $t_{21}$, $t_{22}$, $t_3$,
    $t_{41}$, and $t_{42}$ be terms such that
    \[\Bc(t_1, \Cons(t_{21}, t_{22})) \EqualsBCF \Bc(t_3, \Cons(t_{41}, t_{42})).\]
    Then $t_{22} \EqualsBCF t_{42}$.
\end{Corollary}

\begin{proof}
    From the proof of \cref{thm:bc-semi-cancel}, we see that
    \begin{align*}
        \Ff(t_{21}, t_1) &\EqualsBCF \Ff(t_{41}, t_3) \\
        \Bc(\Ff(t_{21}, t_1), t_{22}) &\EqualsBCF \Bc(\Ff(t_{41}, t_3), t_{42}).
    \end{align*}
    Thus, by the conditional semicancellability of $\Bc$, we see that $t_{22}
    \EqualsBCF t_{42}$.
\end{proof}

\begin{Theorem}\label{thm:bc-eq-cases}
    Let $\FF$ be an element theory, and let $t_1$, $t_2$, $t_3$, and $t_4$ be
    terms such that $\Bc(t_1, t_2) \EqualsBCF \Bc(t_3, t_4)$. Then at least one
    of the following statements is true:
    \begin{enumerate}[(i)]
        \item $t_1 \EqualsBCF t_3$
        \item $t_2 \EqualsBCF \Nil$ and $t_4 \EqualsBCF \Nil$
        \item $t_2 \EqualsBCF \Cons(t_{21}, t_{22})$, $t_4 \EqualsBCF
            \Cons(t_{41}, t_{42})$, and $t_{22} \EqualsBCF t_{42}$, for
            some terms $t_{21}$, $t_{22}$, $t_{41}$, and $t_{42}$.
    \end{enumerate}
\end{Theorem}

\begin{proof}
    Suppose statements (i) and (ii) are false. By
    \cref{lemma:bc-listless-cancel}, $t_2$ and $t_4$ must contain at least one
    occurrence of either $\Cons$ or $\Nil$. Since they are nonnil by
    assumption, they must contain at least one occurrence of $\Cons$. Thus,
    \begin{align*}
        t_2 &\EqualsBCF \Cons(t_{21}, t_{22}) \\
        t_4 &\EqualsBCF \Cons(t_{41}, t_{42})
    \end{align*}
    for some terms $t_{21}$, $t_{22}$, $t_{41}$, and $t_{42}$, and by
    \cref{cor:bc-eq-tail}, $t_{22} \EqualsBCF t_{42}$.

    % Suppose statements (i) and (iii) are false. Again, by
    % \cref{lemma:bc-listless-cancel}, $t_2$ and $t_4$ must contain at least one
    % occurrence of either $\Cons$ or $\Nil$. By \cref{thm:bc-semi-cancel},
    % either $t_2 \not\EqualsBCF t_4$, or $t_2$ and $t_4$ are both equivalent
    % to $\Nil$. If $t_2 \not\EqualsBCF t_4$, they must contain at least one
    % occurrence of $\Cons$. In that case,
    % \begin{align*}
    %     t_2 &\EqualsBCF \Cons(t_{21}, t_{22}) \\
    %     t_4 &\EqualsBCF \Cons(t_{41}, t_{42})
    % \end{align*}
    % for some terms $t_{21}$, $t_{22}$, $t_{41}$, and $t_{42}$. However, our
    % assumption that statement (iii) is false conflicts with
    % \cref{cor:bc-eq-tail}. Thus it must be that $t_2 \EqualsBCF \Nil$ and $t_4
    % \EqualsBCF \Nil$.

    Suppose then that all three statements are false. Since statements (i) and
    (ii) are false, statement (iii) must be true. This is a contradiction, thus
    at least one statement must be true.
\end{proof}

