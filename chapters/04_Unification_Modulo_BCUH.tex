\chapter{Unification Modulo $(\BCUF)$}\label{chap:unif-bcuh}

In this chapter we study the unification problem modulo $(\BCUF)$, where $\FF$
is an element theory. We present an algorithm for solving this problem, and
show that this algorithm is sound, complete, and terminating. In doing so show
that the unification problem modulo $(\BCUF)$ is finitary.

\section{$\BC$-Dependency Graph}

Recall that for a unification problem $\Prob$, we can define the dependency
graph $\DepGraph(\Prob)$ whose nodes are the variables in $\Prob$ and whose
edges are defined by the equations of $\Prob$. When $\Prob$ is a
$(\BCUF)$-unification problem, we define several relations on the variables
of $\Prob$.

\begin{Definition}
    Let $\Prob$ be a $(\BCUF)$-unification problem, and let $\Vl{u}$ and
    $\Vl{v}$ be variables in $\Var(\Prob)$. Then we define the following
    relations on the list variables of $\Prob$:
    \begin{enumerate}[(i)]
        \item $\Vl{u} \Gt{\Cons} \Vl{v}$ if and only if there is an edge from
            $\Vl{u}$ to $\Vl{v}$ in $\DepGraph(\Prob)$ labelled $(\Cons, 2)$.

        \item $\Vl{u} \Gt{\Bc} \Vl{v}$ if and only if there is an edge from
            $\Vl{u}$ to $\Vl{v}$ in $\DepGraph(\Prob)$ labelled $(\Bc, 2)$.

        \item ${\Sym{\Bc}} := {\Gt{\Bc}} \cup {\Lt{\Bc}}$
    \end{enumerate}
\end{Definition}

\begin{Definition}
    Let $\Prob$ be a $(\BCUF)$-unification problem. We define the following
    relation on the list variables of $\Prob$:
    \[ {\Gt{\Len}} := {\Sym[*]{\Bc}} \Compose {\Gt{\Cons}} \Compose
       {({\Gt{\Cons}} \cup {\Sym{\Bc}})^*} \]
    In other words, if $\Vl{u}$ and $\Vl{v}$ are variables in $\Var(\Prob)$,
    then $\Vl{u} \Gt{\Len} \Vl{v}$ if and only if $\Vl{u}$ is ``strictly longer
    than'' $\Vl{v}$.
\end{Definition}

\section{The $\Nonnil$ Set}

Some of the inferences in our algorithm depend on knowing if a variable could
be assigned $\Nil$ in a solution or not. Given a unification problem $\Prob$,
we compute a set $\Nonnil$ of variables that cannot be assigned $\Nil$ in
\emph{any} unifier of $\Prob$.

\begin{Definition}
    Let $\Prob$ be a $(\BCUF)$-unification problem. We define the set $\Nonnil$
    to be the smallest subset of $\Var(\Prob)$ such that, for any list variables
    $\Vl{u}$ and $\Vl{v}$:
    \begin{enumerate}[(i)]
        \item If $\Vl{u} \Gt{\Cons} \Vl{v}$ then $\Vl{u} \in \Nonnil$.
        \item If $\Vl{u} \Sym{\Bc} \Vl{v}$ then $\Vl{u} \in \Nonnil$ if and only
            if $\Vl{v} \in \Nonnil$.
    \end{enumerate}

    Since $\Nonnil$ is the smallest subset with these properties, and since the
    properties only mention list variables, $\Nonnil$ will only contain list
    variables.
\end{Definition}

\section{Inference System $\INF{\BC}$}\label{sec:inf-bc}

Our algorithm is given by a set of inference rules $\INF{\BC}$. The goal is to
transform the list equations of a unification problem into dag-solved form. The
inference rules come in two flavors: \emph{deterministic} and
\emph{nondeterministic}, where determinism means that no backtracking is
required in the rule's application. Rules whose label contains the
\Inf{-nondet} suffix are nondeterministic.

To simplify the definitions of the inference rules, we assume that any relation
defined in terms of a unification problem (e.g., the condition of rule
\ref{inf:nil-soln-3}) refers to the problem in the premise of the rule. We
split $\INF\BC$ into three sections: General list inference rules (L1--L3),
$\BC$-specific inference rules (BC1--BC9), and failure inference rules
(F1--F2).

\subsection{List Inference Rules}

First are the general list inference rules. These rules are all deterministic.
Inference rules \ref{inf:triv-elim} and \ref{inf:var-elim} are the only rules
in $\INF\BC$ that potentially modify element equations --- since the variables
are untyped, these rules can apply to either list or element equations.

\begin{enumerate}[(L1), ref=L\arabic*, align=left]
    \item \InfLabel{triv-elim}{Trivial Elimination}
        \[\InfDef
            { \Prob  }{ u \ueq u }
            { \Prob }{}
        \]
    \item \InfLabel{var-elim}{Variable Elimination}
        \[\CondInfDef
            {       \Prob  }{ u \ueq v }
            { [v/u](\Prob) }{ u \ueq v }
            { if $u \in \Var(\Prob)$ }
        \]

    \item \InfLabel{cancel-cons}{Cancel{}lation on $\Cons$}
        \[\InfDef
            { \Prob }{ \Vl{u} \ueq \Cons(\Ve{v}, \Vl{w}), \;
                       \Vl{u} \ueq \Cons(\Ve{x}, \Vl{y}) }
            { \Prob }{ \Vl{u} \ueq \Cons(\Ve{v}, \Vl{w}), \;
                       \Ve{x} \ueq \Ve{v}, \; \Vl{y} \ueq \Vl{w} }
        \]
\end{enumerate}

\subsection{$\BC$-Specific Inference Rules}

Next are the $\BC$-specific rules. Note that these rules are applied regardless
of the element theory. The last three are nondeterministic.

\begin{enumerate}[(BC1), ref=BC\arabic*, align=left]
    \item \InfLabel{nil-soln-1}{Nil Solution 1}
        \[\InfDef
            { \Prob }{ \Vl{u} \ueq \Bc(\Ve{v}, \Vl{w}), \; \Vl{u} \ueq \Nil }
            { \Prob }{ \Vl{u} \ueq \Nil, \; \Vl{w} \ueq \Nil }
        \]

    \item \InfLabel{nil-soln-2}{Nil Solution 2}
        \[\InfDef
            { \Prob }{ \Vl{u} \ueq \Bc(\Ve{v}, \Vl{w}), \; \Vl{w} \ueq \Nil }
            { \Prob }{ \Vl{u} \ueq \Nil, \; \Vl{w} \ueq \Nil }
        \]

    \item \InfLabel{nil-soln-3}{Nil Solution 3}
        \[\CondInfDef
            { \Prob }{ \Vl{u} \ueq \Bc(\Ve{v}, \Vl{w}) }
            { \Prob }{ \Vl{u} \ueq \Nil, \; \Vl{w} \ueq \Nil }
            {if $\Vl{w} \Gt[+]{\Bc} \Vl{u}$}
        \]

    \item \InfLabel{semi-cancel-bc}{Semi-Cancel{}lation on $\Bc$, at a $\Bc/\Bc$-peak}
        \[\InfDef
            { \Prob }{ \Vl{u} \ueq \Bc(\Ve{v}, \Vl{w}), \; \Vl{u} \ueq \Bc(\Ve{v}, \Vl{x}) }
            { \Prob }{ \Vl{u} \ueq \Bc(\Ve{v}, \Vl{w}), \; \Vl{w} \ueq \Vl{x} }
        \]

    \item \InfLabel{push-bc-below}{Push $\Bc$ Below $\Cons$, at a $\Nonnil$ $\Bc/\Bc$-peak}
        \[\CondInfDef
            { \Prob }{ \Vl{u} \ueq \Bc(\Ve{v}, \Vl{w}), \; \Vl{u} \ueq \Bc(\Ve{x}, \Vl{y}) }
            { \Prob }{ \Vl{w} \ueq \Cons(\Ve{w}', \Vl{z}), \; \Vl{y} \ueq \Cons(\Ve{y}', \Vl{z}), \;
                       \Vl{u} \ueq \Cons(\Ve{u}', \Vl{u}''), \\[4pt]
                       \Vl{u}'' \ueq \Bc(\Ve{u}', \Vl{z}), \; \Ve{u}' \ueq \Ff(\Ve{w}', \Ve{v}), \;
                       \Ve{u}' \ueq \Ff(\Ve{y}', \Ve{x}) }
            { if $\Vl{u} \in \Nonnil$ }
        \]

    \item \InfLabel{splitting}{Splitting, at a $\Cons/\Bc$-peak}
        \[\InfDef
            { \Prob }{ \Vl{u} \ueq \Cons(\Ve{v}, \Vl{w}), \; \Vl{u} \ueq \Bc(\Ve{x}, \Vl{y}) }
            { \Prob }{ \Vl{u} \ueq \Cons(\Ve{v}, \Vl{w}) , \; \Vl{y} \ueq \Cons(\Ve{y}', \Vl{y}''), \;
                       \Vl{w} \ueq \Bc(\Ve{v}, \Vl{y}''), \; \Ve{v} \ueq \Ff(\Ve{y}', \Ve{x}) }
        \]

    \item \InfLabel{nil-soln-nondet}{Nil-solution-Branch for $\Bc$, at a $\Bc/\Bc$-peak}
        \[\InfDef
            { \Prob }{ \Vl{u} \ueq \Bc(\Ve{v}, \Vl{w}), \; \Vl{u} \ueq \Bc(\Ve{x}, \Vl{y}) }
            { \Prob }{ \Vl{u} \ueq \Nil, \; \Vl{w} \ueq \Nil, \; \Vl{y} \ueq \Nil }
        \]

    \item \InfLabel{non-nil-nondet}{Guess a non-nil branch for $\Bc$, at a $\Bc/\Bc$-peak}
        \[\InfDef
            { \Prob }{ \Vl{u} \ueq \Bc(\Ve{v}, \Vl{w}), \; \Vl{u} \ueq \Bc(\Ve{x}, \Vl{y}) }
            { \Prob }{ \Vl{w} \ueq \Cons(\Ve{w}', \Vl{z}), \; \Vl{y} \ueq \Cons(\Ve{y}', \Vl{z}), \;
                       \Vl{u} \ueq \Cons(\Ve{u}', \Vl{u}''), \\[4pt]
                       \Vl{u}'' \ueq \Bc(\Ve{u}', \Vl{z}), \; \Ve{u}' \ueq \Ff(\Ve{w}', \Ve{v}), \;
                       \Ve{u}' \ueq \Ff(\Ve{y}', \Ve{x}) }
        \]

    \item \InfLabel{cancel-bc-nondet}{Cancel{}lation on $\Bc$}
        \[\InfDef
            { \Prob }{ \Vl{u} \ueq \Bc(\Ve{v}, \Vl{w}), \; \Vl{u} \ueq \Bc(\Ve{x}, \Vl{y}) }
            { \Prob }{ \Vl{u} \ueq \Bc(\Ve{v}, \Vl{w}), \; \Ve{x} \ueq \Ve{v}, \; \Vl{y} \ueq \Vl{w} }
        \]
\end{enumerate}

\subsection{Failure Inference Rules}

Lastly we have two failure inference rules. Both are deterministic. Here
$\Fail$ represents an unsolvable unification problem --- essentially a bottom
element.

\begin{enumerate}[(F1), ref=F\arabic*, align=left]
    \item \InfLabel{occurs-check}{Occurs-Check Violation}
        \[\CondInfDef
            { \Prob }{}
            { \Fail }{}
            { if $\Vl{u} \in \Var(\Prob)$ and $\Vl{u} \Gt{\Len} \Vl{u}$ }
        \]

    \item \InfLabel{size-conflict}{Size Conflict}
        \[\InfDef
            { \Prob }{ \Vl{u} \ueq \Cons(\Ve{v}, \Vl{w}), \; \Vl{u} \ueq \Nil }
            { \Fail }{}
        \]
\end{enumerate}

\section{Strategy for Chosing Rules}

The inference rules of $\INF\BC$ should be applied in a particular order. We
provide one strategy, and show that this strategy is terminating.

\begin{Definition}
    Let $\Prob$ be a $(\BCUF)$-unification problem. We define a strategy
    $S_\BC$ for $\INF\BC$ as follows:
    \begin{enumerate}[(1)]
        \item If one of F1 or F2 is applicable, pick that rule.
        \item Otherwise, if one of L1--L3 is applicable, pick the lowest
            numbered applicable rule.
        \item Otherwise, if one of BC1--BC9 is applicable, pick the lowest
            numbered applicable rule.
        \item Otherwise terminate.
    \end{enumerate}
\end{Definition}

To prove termination of $S_\BC$, we first define two new relations on list
variables.

\begin{Definition}
    Let $\Prob$ be a $(\BCUF)$-unification problem, and let $\Vl{u}$ and
    $\Vl{v}$ be variables in $\Var(\Prob)$. We define $\Sym{\EqLabel}$ to be
    the smallest equivalence relation on the list variables of $\Prob$ such
    that $\Vl{u} \Sym{\EqLabel} \Vl{v}$ if either $(\Vl{u} \ueq \Vl{v}) \in
    \Prob$ or $\{\Vl{u} \ueq \Nil, \, \Vl{v} \ueq \Nil\} \subseteq \Prob$.
\end{Definition}

Variables are related by $\Sym{\EqLabel}$ are equivalent in the sense that they
must be assigned equal values in any solution.

\begin{Definition}
    Let $\Prob$ be a $(\BCUF)$-unification problem. We define $\Sym{\beta}$ to
    be the smallest equivalence relation on the list variables of $\Prob$ such
    that:
    \begin{enumerate}[(i)]
        \item ${\Sym{\Bc}} \cup {\Sym{\EqLabel}} \; \subseteq \; {\Sym{\beta}}$
        \item For $\Vl{u}$, $\Vl{v}$, $\Vl{x}$, and $\Vl{y}$ in $\Var(\Prob)$,
            if $\Vl{u} \Gt{\Cons} \Vl{x}$ and $\Vl{v} \Gt{\Cons} \Vl{y}$, then
            $\Vl{u} \Sym{\beta} \Vl{v}$ if and only if $\Vl{x} \Sym{\beta}
            \Vl{y}$.
    \end{enumerate}

    We write $\Var(\Prob)/{\Sym{\beta}}$ for the set of
    $\Sym{\beta}$-equivalence classes of $\Var(\Prob)$, and we write
    $\BetaEqC{\Vl{u}}$ for the equivalence class containing the variable
    $\Vl{u}$.
\end{Definition}

The $\Sym{\beta}$-equivalence classes of $\Prob$ correspond to ``levels'' in
the dependency graph, with $\Bc$-labelled edges running between nodes at the
same level on $\Cons$ ``chains''. We will now make these ideas more formal.

\begin{Definition}
    Let $\Prob$ be a $(\BCUF)$-unification problem. We extend the definition of
    the $\Gt{\Cons}$ relation to $\Sym{\beta}$-equivalence classes so that
    $\EqClass{\Vl{u}}{\beta} \Gt{\Cons} \EqClass{\Vl{v}}{\beta}$ if and only if
    there are variables $\Vl{u}$ and $\Vl{v}$ in $\EqClass{\Vl{u}}{\beta}$ and
    $\EqClass{\Vl{v}}{\beta}$, respectively, such that $\Vl{u} \Gt{\Cons}
    \Vl{v}$.
\end{Definition}

\begin{Lemma}\label{lemma:equiv-class-bound}
    Let $\Prob$ be a $(\BCUF)$-unification problem. For any problem $\Prob'$
    such that $\Prob \InfTo[*]{\INF\BC} \Prob'$,
    \[ 0 \; \leq \;
       |\Var(\Prob') / {\Sym{\beta}}| - |\Var(\Prob) / {\Sym{\beta}}| \; \leq \;
       |\Var(\Prob) / {\Sym{\beta}}| \cdot |\Var(\Prob)| \]
\end{Lemma}

\begin{proof}
    The inference rules of $\INF\BC$ preserve $\Sym{\beta}$-equivalence classes,
    in the sense that if $\Vl{u}$ and $\Vl{v}$ are variables in $\Prob$, then
    $\Vl{u} \Sym{\beta} \Vl{v}$ in $\Prob$ if and only if $\Vl{u} \Sym{\beta}
    \Vl{v}$ in $\Prob'$. Therefore, no equivalence classes are ``lost''; only
    new equivalence classes are added.
    The only rules in $\INF\BC$ that increase the number of $\Cons$ edges, and
    thus potentially the number of $\Sym{\beta}$-equivalence classes, are
    \ref{inf:push-bc-below}, \ref{inf:splitting}, and \ref{inf:non-nil-nondet}.

    In the first case, since $\Vl{u}$ is in $\Nonnil$, $\Vl{u}$ and $\Vl{u}''$
    must be elements of existing $\Sym{\beta}$-equivalence classes. Since
    $\Vl{u}''$ and $\Vl{z}$ share an equivalence class, so do $\Vl{u}$,
    $\Vl{w}$, and $\Vl{y}$. Thus the number of equivalence classes does not
    increase.

    In the second case, since $\Vl{w}$ and $\Vl{y}''$ are in the same
    $\Sym{\beta}$-equivalence class, so are $\Vl{u}$ and $\Vl{y}$. Thus the
    number of equivalence classes does not increase.

    The third case is almost the same as the first, except it is not required
    that $\Vl{u}$ be in $\Nonnil$. Thus the variable $\Vl{u}''$ may be in a new
    $\Sym{\beta}$-equivalence class. However, since only two variables are
    added to the new equivalence class, $|\BetaEqC{\Vl{u}}| >
    |\BetaEqC{\Vl{u}''}|$. Thus the number of new equivalence classes that can
    be created below $\BetaEqC{\Vl{u}}$ is bounded by $|\Var(\Prob)|$, and the
    total number of new equivalence classes is bounded by $|\Var(\Prob) /
    {\Sym{\beta}}| \cdot |\Var(\Prob)|$.
\end{proof}

\begin{Lemma}\label{lemma:gt-cons-partial-order}
    Let $\Prob$ be a $(\BCUF)$-unification problem such that none of the
    failure rules of $\INF\BC$ is applicable. Then $\Gt[+]{\Cons}$ is a
    wel{}l-founded strict partial order on the $\Sym{\beta}$-equivalence
    classes of $\Var(\Prob)$.
\end{Lemma}

\begin{proof}
    The relation $\Gt[+]{\Cons}$ is by its definition transitive. Since failure
    rule \ref{inf:occurs-check} does not apply, $\Gt[+]{\Cons}$ is also
    irreflexive. Thus it is asymmetric and we need only show well-foundedness.

    By \cref{lemma:equiv-class-bound} the number of $\Sym{\beta}$-equivalence
    classes is finite, so if $\Gt[+]{\Cons}$ were not well-founded, there would
    have to be an equivalence class $\EqClass{\Vl{u}}{\beta}$ such that
    $\EqClass{\Vl{u}}{\beta} \Gt[+]{\Cons} \EqClass{\Vl{u}}{\beta}$. But then
    there would be a variable $\Vl{u} \in \EqClass{\Vl{u}}{\beta}$ such that
    $\Vl{u} \Gt{\Len} \Vl{u}$ and \ref{inf:occurs-check} would apply. Thus
    $\Gt[+]{\Cons}$ is well-founded on $\Sym{\beta}$-equivalence classes.
\end{proof}

\begin{Lemma}\label{lemma:total-connected-comp}
    Let $\Prob$ be a $(\BCUF)$-unification problem such that $\Gt[+]{\Cons}$ is
    a strict partial order on the $\Sym{\beta}$-equivalence classes of
    $\Var(\Prob)$. Let $\DepGraph(\Prob)$ be the dependency graph of $\Prob$,
    and let $C$ be a connected component of $\DepGraph(\Prob)$. Then
    $\Gt[+]{\Cons}$ is total on the $\Sym{\beta}$-equivalence classes of nodes
    in $C$.
\end{Lemma}

\begin{proof}
    Consider two $\Gt[+]{\Cons}$-incomparable $\Sym{\beta}$-equivalence classes
    $\BetaEqC{\Vl{u}}$ and $\BetaEqC{\Vl{v}}$ of nodes in $C$. Since these
    equivalence classes are in the same connected component, there must be a
    node $\Vl{w}$ such that there is a path to or from $\Vl{w}$ to some
    $\Vl{u}$ in $\BetaEqC{\Vl{u}}$ and some $\Vl{v}$ in $\BetaEqC{\Vl{v}}$. If
    there is a path from $\Vl{u}$ to $\Vl{w}$ to $\Vl{v}$ or from $\Vl{v}$ to
    $\Vl{w}$ to $\Vl{u}$, then $\EqClass{\Vl{u}}{\beta} \Gt[+]{\Cons}
    \EqClass{\Vl{v}}{\beta}$ or $\EqClass{\Vl{v}}{\beta} \Gt[+]{\Cons}
    \EqClass{\Vl{u}}{\beta}$, respectively. So, if $\BetaEqC{\Vl{u}}$ and
    $\BetaEqC{\Vl{v}}$ are incomparable, then either $\BetaEqC{\Vl{u}}
    \Gt[+]{\Cons} \BetaEqC{\Vl{w}} \Lt[+]{\Cons} \BetaEqC{\Vl{v}}$, or
    $\BetaEqC{\Vl{u}} \Lt[+]{\Cons} \BetaEqC{\Vl{w}} \Gt[+]{\Cons}
    \BetaEqC{\Vl{v}}$.

    In the first case when $\BetaEqC{\Vl{u}} \Gt[+]{\Cons} \BetaEqC{\Vl{w}}
    \Lt[+]{\Cons} \BetaEqC{\Vl{v}}$, there must be variables $\Vl{w}_1$ and
    $\Vl{w}_2$ in $\BetaEqC{\Vl{w}}$ such that $\Vl{u} \Gt[+]{\Cons} \Vl{w}_1
    \Sym{\beta} \Vl{w}_2 \Lt[+]{\Cons} \Vl{v}$. But then $\Vl{u} \Sym{\beta}
    \Vl{v}$ and $\BetaEqC{\Vl{u}} = \BetaEqC{\Vl{v}}$. Similarly, in the second
    case when $\BetaEqC{\Vl{u}} \Gt[+]{\Cons} \BetaEqC{\Vl{w}} \Lt[+]{\Cons}
    \BetaEqC{\Vl{v}}$, there must be variables $\Vl{w}_1$ and $\Vl{w}_2$ in
    $\BetaEqC{\Vl{w}}$ such that $\Vl{u} \Lt[+]{\Cons} \Vl{w}_1 \Sym{\beta}
    \Vl{w}_2 \Gt[+]{\Cons} \Vl{v}$. But again, then $\Vl{u} \Sym{\beta} \Vl{v}$
    and $\BetaEqC{\Vl{u}} = \BetaEqC{\Vl{v}}$.

    Thus, if $\BetaEqC{\Vl{u}}$ and $\BetaEqC{\Vl{v}}$ are incomparable in
    $\Gt[+]{\Cons}$, then $\BetaEqC{\Vl{u}} = \BetaEqC{\Vl{v}}$. Therefore,
    $\Gt[+]{\Cons}$ is a strict total order on $\Sym{\beta}$-equivalence
    classes of nodes in $C$. \end{proof}

\begin{Definition}
    Let $\Prob$ be a $(\BCUF)$-unification problem such that $\Gt[+]{\Cons}$ is
    a well-founded strict partial order on the $\Sym{\beta}$-equivalence
    classes of $\Var(\Prob)$. Let $\EqClass{\Vl{u}}{\beta}$ be a
    $\Sym{\beta}$-equivalence class. We define the $\Cons$-depth of
    $\EqClass{\Vl{u}}{\beta}$, written $\ConsDepth(\EqClass{\Vl{u}}{\beta})$, as
    follows:
    \begin{enumerate}[(i)]
        \item If $\EqClass{\Vl{u}}{\beta}$ is maximal in $\Gt[+]{\Cons}$, then
            $\ConsDepth(\EqClass{\Vl{u}}{\beta}) := 0$.
        \item Otherwise, $\ConsDepth(\EqClass{\Vl{u}}{\beta}) :=
            \ConsDepth(\EqClass{\Vl{v}}{\beta}) + 1$, where $\EqClass{\Vl{v}}{\beta}
            \Gt{\Cons} \EqClass{\Vl{u}}{\beta}$.
    \end{enumerate}
    Note that, in the second case, $\EqClass{\Vl{v}}{\beta}$ must be unique by
    \cref{lemma:total-connected-comp}.

    For convenience, we extend this definition to list variables in the obvious
    way, so that $\ConsDepth(\Vl{u}) := \ConsDepth(\EqClass{\Vl{u}}{\beta})$.
\end{Definition}

\begin{Theorem}\label{lemma:inf-bc-terminates}
    Let $\Prob$ be a $(\BCUF)$-unification problem. Applying inference rules
    from $\INF\BC$ using strategy $S_\BC$ always terminates.
\end{Theorem}

\begin{proof}
    Suppose the strategy does not terminate for some initial problem $\Prob_0$.
    Since there are a finite number of inference rules, at least one inference
    rule must be applied infinitely many times. Clearly this rule must not be a
    failure rule, since those rules cause immediate termination.

    For the remaining rules, consider the measure function
    $\phi\colon \ProbType \to \Nat^3$ given by:
    \begin{align*}
        \phi(\Prob) &:= \left(
            \text{\textit{free-classes}}, \;
            \text{\textit{free-depth}}, \;
            |\DepGraph(\Prob)|, \;
            |\Prob|
        \right)
        \intertext{where}
        \text{\textit{free-classes}} &:= m - |\Var(\Prob)/{\Sym{\beta}}| \\[4pt]
        \text{\textit{free-depth}} &:= \sum \, \{
            m - \ConsDepth(\Vl{u}) \mid (\Vl{u} \ueq \Bc(\Ve{v}, \Vl{w})) \in \Prob
        \} \\[4pt]
        m &:= |\Var(\Prob_0)/{\Sym{\beta}}| \cdot (1 + |\Var(\Prob_0)|)
    \end{align*}

    The first component, \textit{free-classes}, measures the number of number
    of unused $\Sym{\beta}$-equivalence class ``slots''. The second component,
    \textit{free-depth}, is intuitively a measure of how many times a
    $\Bc$-edge can be ``lowered'' by rules like \ref{inf:splitting}. Note that
    $m$, which measures the maximum number of $\Sym{\beta}$-equivalence
    classes, is fixed by the initial input probem $\Prob_0$ thanks to the bound
    given by \cref{lemma:equiv-class-bound}.

    The function $\phi$ is only defined when $\ConsDepth$ is defined --- i.e.,
    only when $\Gt[+]{\Cons}$ is a well-founded strict partial order on the
    $\Sym{\beta}$-equivalence classes. Since the failure rules must not apply,
    this is the case according to \cref{lemma:gt-cons-partial-order}.

    For each remaining inference rule in $\INF\BC$, we will show that applying
    the rule decreases the measure of the problem.

    \begin{itemize}[align=left]
        \item[(\ref{inf:triv-elim})] This rule preserves the first three
            components and decreases the last.

        \item[(\ref{inf:var-elim}--\ref{inf:cancel-cons})] These rules preserve
            or decrease the first two components and decrease the third.
    \end{itemize}

    \begin{itemize}[align=left]
        \item[(\ref{inf:nil-soln-1}--\ref{inf:semi-cancel-bc})] These rules all
            preserve or decrease the first two components and decrease the
            third.

        \item[(\ref{inf:push-bc-below}--\ref{inf:splitting})] While these rules
            do add several new variables and edges to the dependency graph,
            they add no new $\Sym{\beta}$-equivalence classes, and they remove
            a $\Bc$-equation and add a new one with a variable that has a
            greater $\Cons$-depth. Thus \ref{inf:push-bc-below} and
            \ref{inf:splitting} preserve the first component and decrease the
            second.

        \item[(\ref{inf:nil-soln-nondet})] This rule behaves the same as
            \ref{inf:nil-soln-1}--\ref{inf:nil-soln-3}, preserving or decreasing
            the first two components and decreasing the third.

        \item[(\ref{inf:non-nil-nondet})] This rule either behaves the same as
            \ref{inf:push-bc-below}, preserving the first component and
            decreasing the second, or it creates a new
            $\Sym{\beta}$-equivalence class and decreases the first component.

        \item[(\ref{inf:cancel-bc-nondet})] This rule behaves the same as
            \ref{inf:semi-cancel-bc}, preserving or decreasing the first two
            components and decreasing the third.
    \end{itemize}

    Each non-failure inference rule in $\INF\BC$ causes the measure of the
    problem to strictly decrease. Since $>$ is well-founded on $\Nat^3$,
    this cannot continue infinitely, so no rule in $\INF\BC$ can be applied
    infinitely many times. Thus there is no problem $\Prob_0$ such that
    strategy $S_\BC$ for $\INF\BC$ does not terminate.
\end{proof}

\section{Algorithm}\label{sec:bc-algorithm}

\section{Soundness}\label{sec:bc-soundness}

\section{Completeness}\label{sec:bc-completeness}

\section{Termination}\label{sec:bc-termination}

\section{Runtime Analysis}\label{sec:bc-runtime-analysis}

\section{Dependency Graph Transformations}\label{sec:dep-graph-trans}

