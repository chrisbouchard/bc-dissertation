\chapter{Unification Modulo $(\BCUF)$}\label{chap:unif-bcuh}

In this chapter we study the unification problem modulo $(\BCUF)$, where $\FF$
is an element theory. We present an algorithm for solving this problem, and
show that this algorithm is sound, complete, and terminating. In doing so show
that the unification problem modulo $(\BCUF)$ is finitary.

In this chapter, and in all subsequent chapters, we will assume that all
unification problems are in standard form as described in \cref{sec:std-form}.
With that in mind, we make the following definition:

\setcounter{Theorem}{0}
\begin{Definition}
    We partition the equations in a $(\BCUF)$-unification problem based on the
    sort of their left-hand variable. We write $\ListEq(\Prob)$ for the list
    equations of $\Prob$, and $\EltEq(\Prob)$ for the element equations.
\end{Definition}

It is worth noting that any unifier of $\Prob$ is a unifier of every subset of
$\Prob$, in particular $\ListEq(\Prob)$ and $\EltEq(\Prob)$.

\section{$\BC$-Dependency Graph}

Recall that for a unification problem $\Prob$, we can define the dependency
graph $\DepGraph(\Prob)$ whose nodes are the variables in $\Prob$ and whose
edges are defined by the equations of $\Prob$. When $\Prob$ is a
$(\BCUF)$-unification problem, we define several relations on the variables of
$\Prob$.

\begin{Definition}
    Let $\Prob$ be a $(\BCUF)$-unification problem, and let $\Vl{u}$ and
    $\Vl{v}$ be variables in $\Var(\Prob)$. Then we define the following
    relations on the list variables of $\Prob$:
    \begin{enumerate}[(i)]
        \item $\Vl{u} \Gt{\Cons} \Vl{v}$ if and only if there is an edge from
            $\Vl{u}$ to $\Vl{v}$ in $\DepGraph(\Prob)$ labelled $(\Cons, 2)$.

        \item $\Vl{u} \Gt{\Bc} \Vl{v}$ if and only if there is an edge from
            $\Vl{u}$ to $\Vl{v}$ in $\DepGraph(\Prob)$ labelled $(\Bc, 2)$.

        \item ${\Sym{\Bc}} := {\Gt{\Bc}} \cup {\Lt{\Bc}}$
    \end{enumerate}
\end{Definition}

\begin{Definition}
    Let $\Prob$ be a $(\BCUF)$-unification problem. We define the following
    relation on the list variables of $\Prob$:
    \[ {\Gt{\Len}} := {\Sym[*]{\Bc}} \Compose {\Gt{\Cons}} \Compose
       {({\Gt{\Cons}} \cup {\Sym{\Bc}})^*} \]
    In other words, if $\Vl{u}$ and $\Vl{v}$ are variables in $\Var(\Prob)$,
    then $\Vl{u} \Gt{\Len} \Vl{v}$ if and only if $\Vl{u}$ is ``strictly longer
    than'' $\Vl{v}$ in every unifier of $\Prob$.
\end{Definition}

\section{The $\Nonnil$ Set}

Some of the inferences in our algorithm depend on knowing if a variable could
be assigned $\Nil$ in a unifier or not. Given a unification problem $\Prob$,
we compute a set $\Nonnil$ of variables that cannot be assigned $\Nil$ in
\emph{any} unifier of $\Prob$.

\begin{Definition}
    Let $\Prob$ be a $(\BCUF)$-unification problem. We define the set $\Nonnil$
    to be the smallest subset of $\Var(\Prob)$ such that, for any list variables
    $\Vl{u}$ and $\Vl{v}$:
    \begin{enumerate}[(i)]
        \item If $\Vl{u} \Gt{\Cons} \Vl{v}$ then $\Vl{u} \in \Nonnil$.
        \item If $\Vl{u} \Sym{\Bc} \Vl{v}$ then $\Vl{u} \in \Nonnil$ if and only
            if $\Vl{v} \in \Nonnil$.
    \end{enumerate}

    Since $\Nonnil$ is the smallest subset with these properties, and since the
    properties only mention list variables, $\Nonnil$ will only contain list
    variables.
\end{Definition}

\section{Inference System $\INF{\BC}$}\label{sec:inf-bc}

Our algorithm is given by an inference system $\INF\BC$. The goal is to
transform the list equations of a unification problem into dag-solved form.
Our system $\INF\BC$ uses both deterministic and nondeterministic inference
rules.

To simplify the definitions of the inference rules, we assume that any relation
defined in terms of a unification problem (e.g., the condition of rule
\InfRef{nil-soln-3}) refers to the problem in the premise of the rule. We
split $\INF\BC$ into three sections: General list inference rules (L1--L3),
$\BC$-specific inference rules (BC1--BC9), and failure inference rules
(F1--F2).

\subsection{List Inference Rules}

First are the general list inference rules. These rules are all deterministic.
Inference rules \InfRef{triv-elim} and \InfRef{var-elim} are the only rules
in $\INF\BC$ that potentially modify element equations --- since the variables
are untyped, these rules can apply to either list or element equations.

\begin{enumerate}[(L1), ref=L\arabic*, align=left]
    \item \InfLabel{triv-elim}{Trivial elimination}
        \[\InfDef
            { \Prob  }{ u \ueq u }
            { \Prob }{}
        \]
    \item \InfLabel{var-elim}{Variable elimination}
        \[\CondInfDef
            {       \Prob  }{ u \ueq v }
            { [v/u](\Prob) }{ u \ueq v }
            { if $u \in \Var(\Prob)$ }
        \]

    \item \InfLabel{cancel-cons}{Cancel{}lation on $\Cons$}
        \[\InfDef
            { \Prob }{ \Vl{u} \ueq \Cons(\Ve{v}, \Vl{w}), \;
                       \Vl{u} \ueq \Cons(\Ve{x}, \Vl{y}) }
            { \Prob }{ \Vl{u} \ueq \Cons(\Ve{v}, \Vl{w}), \;
                       \Ve{x} \ueq \Ve{v}, \; \Vl{y} \ueq \Vl{w} }
        \]
\end{enumerate}

\subsection{$\BC$-Specific Inference Rules}

Next are the $\BC$-specific rules. Note that these rules are applied regardless
of the element theory. The last three are nondeterministic.

\begin{enumerate}[(BC1), ref=BC\arabic*, align=left]
    \item \InfLabel{nil-soln-1}{Nil solution 1}
        \[\InfDef
            { \Prob }{ \Vl{u} \ueq \Bc(\Ve{v}, \Vl{w}), \;
                       \Vl{u} \ueq \Nil }
            { \Prob }{ \Vl{u} \ueq \Nil, \;
                       \Vl{w} \ueq \Nil }
        \]

    \item \InfLabel{nil-soln-2}{Nil solution 2}
        \[\InfDef
            { \Prob }{ \Vl{u} \ueq \Bc(\Ve{v}, \Vl{w}), \;
                       \Vl{w} \ueq \Nil }
            { \Prob }{ \Vl{u} \ueq \Nil, \;
                       \Vl{w} \ueq \Nil }
        \]

    \item \InfLabel{nil-soln-3}{Nil solution 3}
        \[\CondInfDef
            { \Prob }{ \Vl{u} \ueq \Bc(\Ve{v}, \Vl{w}) }
            { \Prob }{ \Vl{u} \ueq \Nil, \; \Vl{w} \ueq \Nil }
            { if $\Vl{w} \Gt[+]{\Bc} \Vl{u}$ }
        \]

    \item \InfLabel{semi-cancel-bc}{Semi-cancel{}lation on $\Bc$}
        \[\InfDef
            { \Prob }{ \Vl{u} \ueq \Bc(\Ve{v}, \Vl{w}), \;
                       \Vl{u} \ueq \Bc(\Ve{v}, \Vl{x}) }
            { \Prob }{ \Vl{u} \ueq \Bc(\Ve{v}, \Vl{w}), \;
                       \Vl{w} \ueq \Vl{x} }
        \]

    \item \InfLabel{push-bc-below}{Push $\Bc$ below $\Cons$}
        \[\CondInfDef
            { \Prob }{ \Vl{u} \ueq \Bc(\Ve{v}, \Vl{w}), \;
                       \Vl{u} \ueq \Bc(\Ve{x}, \Vl{y}) }
            { \Prob }{ \Vl{w} \ueq \Cons(\Ve{w}', \Vl{z}), \;
                       \Vl{y} \ueq \Cons(\Ve{y}', \Vl{z}), \;
                       \Vl{u} \ueq \Cons(\Ve{u}', \Vl{u}''), \\[4pt]
                       \Vl{u}'' \ueq \Bc(\Ve{u}', \Vl{z}), \;
                       \Ve{u}' \ueq \Ff(\Ve{w}', \Ve{v}), \;
                       \Ve{u}' \ueq \Ff(\Ve{y}', \Ve{x}) }
            { if $\Vl{u} \in \Nonnil$ }
        \]

    \item \InfLabel{splitting}{Splitting}
        \[\InfDef
            { \Prob }{ \Vl{u} \ueq \Cons(\Ve{v}, \Vl{w}), \;
                       \Vl{u} \ueq \Bc(\Ve{x}, \Vl{y}) }
            { \Prob }{ \Vl{u} \ueq \Cons(\Ve{v}, \Vl{w}) , \;
                       \Vl{y} \ueq \Cons(\Ve{y}', \Vl{y}''), \;
                       \Vl{w} \ueq \Bc(\Ve{v}, \Vl{y}''), \;
                       \Ve{v} \ueq \Ff(\Ve{y}', \Ve{x}) }
        \]

    \item \InfLabel{nil-soln-nondet}{Nil branch (Nondeterministic)}
        \[\InfDef
            { \Prob }{ \Vl{u} \ueq \Bc(\Ve{v}, \Vl{w}), \;
                       \Vl{u} \ueq \Bc(\Ve{x}, \Vl{y}) }
            { \Prob }{ \Vl{u} \ueq \Nil, \; \Vl{w} \ueq \Nil, \;
                       \Vl{y} \ueq \Nil }
        \]

    \item \InfLabel{non-nil-nondet}{Non-nil branch (Nondeterministic)}
        \[\InfDef
            { \Prob }{ \Vl{u} \ueq \Bc(\Ve{v}, \Vl{w}), \;
                       \Vl{u} \ueq \Bc(\Ve{x}, \Vl{y}) }
            { \Prob }{ \Vl{w} \ueq \Cons(\Ve{w}', \Vl{z}), \;
                       \Vl{y} \ueq \Cons(\Ve{y}', \Vl{z}), \;
                       \Vl{u} \ueq \Cons(\Ve{u}', \Vl{u}''), \\[4pt]
                       \Vl{u}'' \ueq \Bc(\Ve{u}', \Vl{z}), \;
                       \Ve{u}' \ueq \Ff(\Ve{w}', \Ve{v}), \;
                       \Ve{u}' \ueq \Ff(\Ve{y}', \Ve{x}) }
        \]

    \item \InfLabel{cancel-bc-nondet}{Cancel{}lation branch (Nondeterministic)}
        \[\InfDef
            { \Prob }{ \Vl{u} \ueq \Bc(\Ve{v}, \Vl{w}), \;
                       \Vl{u} \ueq \Bc(\Ve{x}, \Vl{y}) }
            { \Prob }{ \Vl{u} \ueq \Bc(\Ve{v}, \Vl{w}), \;
                       \Ve{x} \ueq \Ve{v}, \; \Vl{y} \ueq \Vl{w} }
        \]
\end{enumerate}

\subsection{Failure Inference Rules}

Lastly we have two failure inference rules. Both are deterministic. Here
$\Fail$ represents an unsolvable unification problem --- essentially a bottom
element.

\begin{enumerate}[(F1), ref=F\arabic*, align=left]
    \item \InfLabel{occurs-check}{Occurs-Check Violation}
        \[\CondInfDef
            { \Prob }{}
            { \Fail }{}
            { if $\Vl{u} \in \Var(\Prob)$ and $\Vl{u} \Gt{\Len} \Vl{u}$ }
        \]

    \item \InfLabel{size-conflict}{Size Conflict}
        \[\InfDef
            { \Prob }{ \Vl{u} \ueq \Cons(\Ve{v}, \Vl{w}), \;
                       \Vl{u} \ueq \Nil }
            { \Fail }{}
        \]
\end{enumerate}



\section{Soundness and Completeness of $\INF\BC$}

\begin{Lemma}\label{lemma:infl-dag-solved}
    Let $\Prob$ be a $(\BCUF)$-unification problem such that $\INF\BC$ is
    not applicable. Then $\ListEq(\Prob)$ is in dag-solved form.
\end{Lemma}

\begin{proof}
    Suppose $\ListEq(\Prob)$ were not in dag-solved form. Then either there
    is a variable that occurs on the left-hand side of two different equations,
    or there is a cycle in the equations such that they cannot be ordered.

    In the fist case, there are a few possibilities for the peak:
    \begin{itemize}[align=left]
        \item[$\Cons/\Cons$:] Inference rule \InfRef{cancel-cons} is
            applicable.
        \item[$\Cons/\Nil$:] Inference rule \InfRef{size-conflict} is
            applicable.
        \item[$\Bc/\Cons$:] Inference rule \InfRef{splitting} is applicable.
        \item[$\Bc/\Nil$:] Inference rule \InfRef{nil-soln-1} is applicable.
        \item[$\Bc/\Bc$:] Either of the deterministic inference rules
            \InfRef{semi-cancel-bc} or \InfRef{push-bc-below}, or one of the
            nondeterministic inference rules
            \InfRef{nil-soln-nondet}--\InfRef{cancel-bc-nondet}, is
            applicable.
        \item[$\Nil/\Nil$:] This case is not possible, since $\Prob$ is a set
            and $\Nil$ is a constant.
    \end{itemize}
    For each possibility an inference rule from $\INF\BC$ is be applicable, so
    there cannot be a peak.

    In the second case, there is a cycle among the list equations such that
    they cannot be ordered. Without loss of generality, we will assume that
    this cycle contains no equations with variable right-hand sides (i.e.,
    equations of the form $\Vl{u} \ueq \Vl{v}$), because otherwise we could
    apply rule \InfRef{var-elim}. If the cycle contains no $\Cons$-equations,
    then we can apply rule \InfRef{nil-soln-3}. If the cycle does contain
    $\Cons$-equations, then we can apply rule \InfRef{occurs-check}. Thus a
    rule from $\INF\BC$ would be applicable, and there cannot be a cycle.

    In both cases, a rule from $\INF\BC$ is applicable to $\Prob$, which is
    a contradiction. Thus $\ListEq(\Prob)$ must be in dag-solved form.
\end{proof}

\begin{Theorem}[Soundness of $\INF\BC$]\label{thm:inf-bc-sound}
    Let $\Prob$ and $\Prob'$ be $(\BCUF)$-unification problems such that $\Prob
    \InfTo[*]{\INF\BC} \Prob'$, and let $\Unifier$ be a unifier of $\Prob'$.
    Then $\Unifier$ is a unifier of $\Prob$.
\end{Theorem}

\begin{proof}
    The soundness of $\INF\BC$ follows from the soundness of all its inference
    rules. Since $\Prob \InfTo[*]{\INF\BC} \Prob'$, there is some $n$ such that
    $\Prob \InfTo[n]{\INF\BC} \Prob'$. We will prove the theorem by induction
    on $n$.

    In the base case when $n = 0$, we have that $\Prob = \Prob'$. So the
    theorem holds trivially.

    In the inductive step, suppose the theorem holds for all $\Prob$ and
    $\Prob'$ such that $\Prob \InfTo[n-1]{\INF\BC} \Prob'$. Since $n > 0$, we
    know there must be some problem $\Prob''$ such that $\Prob \InfTo{\INF\BC}
    \Prob'' \InfTo[n-1]{\INF\BC} \Prob'$, and by our inductive hypothesis,
    $\Unifier$ is a unifier of $\Prob''$. Since all the inference rules of
    $\INF\BC$ are sound, $\Unifier$ must be a unifier of $\Prob$ and the
    theorem is proven.
\end{proof}

\begin{Lemma}\label{lemma:inf-bc-complete}
    Let $\Prob$ be a $(\BCUF)$-unification problem with a peak at a list
    variable, and let $\Unifier$ be a unifier of $\Prob$. Then $\Prob
    \InfTo{\INF\BC} \Prob'$ such that $\Unifier$ is a unifier of $\Prob'$.
\end{Lemma}

\begin{proof}
    If there is a peak at a list variable, there are three cases:
    $\Cons/\Cons$, $\Cons/\Bc$, or $\Bc/\Bc$. In the first two cases, there are
    deterministic rules that always apply: \InfRef{cancel-cons} and
    \InfRef{splitting}, respectively. If there is a deterministic rule in
    $\INF\BC$ that is applicable, then the proof follows from the fact that
    deterministic rules preserve the set of unifiers. So we assume that no
    deterministic rule is applicable, and the only remaining case is a
    $\Bc/\Bc$-peak.

    If $\Prob$ has a $\Bc/\Bc$-peak, then $\Prob = \Prob'' \uplus \{ \Vl{u}
    \ueq \Bc(\Ve{v}, \Vl{w}), \, \Vl{u} \ueq \Bc(\Ve{x}, \Vl{y}) \}$ for some
    variables $\Vl{u}$, $\Ve{v}$, $\Vl{w}$, $\Ve{x}$, and $\Vl{y}$, and some
    subproblem $\Prob''$. So,
    \[ \Unifier(\Bc(\Ve{v}, \Vl{w})) \,=\, \Bc(\Unifier(\Ve{v}), \Unifier(\Vl{w}))
    \,\EqualsBCF\, \Unifier(\Vl{u}) \,\EqualsBCF\, \Bc(\Unifier(\Ve{x}),
    \Unifier(\Vl{y})) \,=\, \Unifier(\Bc(\Ve{x}, \Vl{y})). \]
    By \cref{cor:bc-preserve-cons}, $\Unifier(\Vl{u})$, $\Unifier(\Vl{w})$, and
    $\Unifier(\Vl{y})$ all have the same length, so we have two cases to
    consdier: either they are all nil, or they are all nonnil.

    If they are all nil, we can apply rule \InfRef{nil-soln-nondet} to obtain
    a problem $\Prob' = \Prob'' \uplus \{ \Vl{u} \ueq \Nil, \, \Vl{w} \ueq
    \Nil, \, \Vl{y} \ueq \Nil \}$. Clearly, then, $\Unifier$ is a unifier of
    $\Prob'$. If they are all nonnil, we can use the conditional
    semicancellativity of $\Bc$ from \cref{thm:bc-semi-cancel} to see that
    $\Unifier(\Ve{v}) \EqualsBCF \Unifier(\Ve{x})$ if and only if
    $\Unifier(\Vl{w}) \EqualsBCF \Unifier(\Vl{y})$. If both equivalences hold,
    then we can apply rule \InfRef{cancel-bc-nondet} to obtain a problem
    $\Prob' = \Prob'' \uplus \{ \Vl{u} \ueq \Bc(\Ve{v}, \Vl{w}), \, \Ve{x} \ueq
    \Ve{v}, \, \Vl{y} \ueq \Vl{w} \}$, and $\Unifier$ is a unifier of $\Prob'$.

    In the remaining case, $\Unifier(\Vl{u})$, $\Unifier(\Vl{w})$, and
    $\Unifier(\Vl{y})$ are all nonnil, $\Unifier(\Ve{v}) \not\EqualsBCF
    \Unifier(\Ve{x})$, and $\Unifier(\Vl{w}) \not\EqualsBCF \Unifier(\Vl{y})$.
    In this case,
    \begin{align*}
        \Unifier(\Vl{u}) &\EqualsBCF \Cons(t_{11}, t_{12}) \\
        \Unifier(\Vl{w}) &\EqualsBCF \Cons(t_{21}, t_{22}) \\
        \Unifier(\Vl{y}) &\EqualsBCF \Cons(t_{31}, t_{32})
    \end{align*}
    By \cref{cor:bc-eq-tail}, we see that $t_{22} \EqualsBCF t_{32}$. We can
    apply rule \InfRef{non-nil-nondet} to obtain a problem $\Prob'$ such that
    \begin{align*} \Prob' \; = \; \Prob'' \uplus \{ \,
        &\Vl{w} \ueq \Cons(\Ve{w}', \Vl{z}), \,
        \Vl{y} \ueq \Cons(\Ve{y}', \Vl{z}), \,
        \Vl{u} \ueq \Cons(\Ve{u}', \Vl{u}''), \\
        &\Vl{u}'' \ueq \Bc(\Ve{u}', \Vl{z}), \,
        \Ve{u}' \ueq \Ff(\Ve{w}', \Ve{v}), \,
        \Ve{u}' \ueq \Ff(\Ve{y}', \Ve{x})
        \, \}
    \end{align*}
    and again, $\Unifier$ is a unifier of $\Prob'$. In all cases $\Unifier$ is
    a unifier of $\Prob'$, and the lemma is proven.
\end{proof}

\begin{Theorem}[Completeness of $\INF\BC$]\label{thm:inf-bc-complete}
    Let $\Prob$ be a $(\BCUF)$-unification problem, and let $\Unifier$ be a
    unifier of $\Prob$. Then $\Prob \InfTo[*]{\INF\BC} \Prob'$ such that
    $\Unifier$ is a unifier of $\Prob'$.
\end{Theorem}

\begin{proof}
    Since the relation $\InfTo{\INF\BC}$ is well-founded, there is a bound $n$
    such that all of the $\InfTo{\INF\BC}$-normal forms of $\Prob$ can be
    reached in $n$ steps or fewer. We will prove this theorem by induction on
    $n$.

    We first consider the base case, when $n = 0$. Thus $\Prob
    \InfTo[*]{\INF\BC} \Prob$ and the theorem holds trivially.

    In the inductive step, we assume the theorem holds for any problem
    $\Prob''$ such that all $\InfTo{\INF\BC}$-normal forms of $\Prob''$ can be
    reached in $n - 1$ steps or fewer. By \cref{lemma:inf-bc-complete}, there
    is such a problem $\Prob''$ such that $\Prob \InfTo{\INF\BC} \Prob''$ and
    $\Unifier$ is a unifier of $\Prob''$. By our inductive hypothesis, there is
    a problem $\Prob'$ such that $\Prob'' \InfTo[*]{\INF\BC} \Prob'$ and
    $\Unifier$ is a unifier of $\Prob'$. Thus $\Prob \InfTo[*]{\INF\BC} \Prob'$
    and the theorem follows.
\end{proof}

\begin{Theorem}\label{thm:inf-bc-mgu-complete}
    Let $\Prob$ and $\Prob'$ be $(\BCUF)$-unification problems such that $\Prob
    \InfTo[*]{\INF\BC} \Prob'$, and let $\Unifier$ be a most general unifier of
    $\Prob$ and a unifier of $\Prob'$. Then $\Unifier$ is a most general
    unifier of $\Prob'$.
\end{Theorem}

\begin{proof}
    Suppose $\Unifier$ were not a most general unifier of $\Prob'$. Then there
    is some other unifier $\Unifier'$ of $\Prob'$ such that $\Unifier
    \LessGeneralThan \Unifier'$. By the soundness of $\INF\BC$ shown in
    \cref{thm:inf-bc-sound}, $\Unifier'$ is also a unifier of $\Prob$. This is
    a contradiction, since $\Unifier$ is a most general unifier of $\Prob$. Thus
    $\Unifier$ is a most general unifier of $\Prob'$.
\end{proof}

% \begin{Lemma}\label{lemma:inf-bc-mgu-sound}
%     Let $\Prob$ and $\Prob'$ be $(\BCUF)$-unification problems such that $\Prob
%     \InfTo{\INF\BC} \Prob'$, and let $\Unifier$ be a most general unifier of
%     $\Prob'$. Then $\Unifier$ is a most general unifier of $\Prob$.
% \end{Lemma}
%
% \begin{proof}
%     Suppose $\Unifier$ is not a most general unifier of $\Prob$. Then there is
%     a unifier $\MGU$ of $\Prob$ such that $\Unifier \LessGeneralThan \MGU$.
%     Since $\Prob \InfTo{\INF\BC} \Prob'$, there is an inference rule $\rho$
%     from $\INF\BC$ such that $\Prob \InfTo{\rho} \Prob'$. If $\rho$ were a
%     deterministic rule, then $\MGU$ would also a unifier of $\Prob'$, which is
%     a contradiction. Thus $\rho$ must be one of the nondeterministic rules
%     \InfRef{nil-soln-nondet}, \InfRef{non-nil-nondet}, or
%     \InfRef{cancel-bc-nondet}.
%
%     Let $\X := \Var(\Unifier) \, \cup \, \Var(\MGU)$. Let $\AltSubst[\rho]$ be
%     the substitution such that matches $\Prob$ to the premise of $\rho$ and
%     $\Prob'$ to its conclusion. Let $\X_\rho := \Var(\AltSubst[\rho])$. We can
%     partition the unifiers $\Unifier$ and $\MGU$ as follows:
%     \begin{align*}
%         \Unifier &= \Sub{\Unifier}{(\X \setminus \X_\rho)} \uplus \Sub{\Unifier}{\X_\rho} \\
%         \MGU &= \Sub{\MGU}{(\X \setminus \X_\rho)} \uplus \Sub{\MGU}{\X_\rho}
%     \end{align*}
%     We can assume without loss of generality that $\Sub{\Unifier}{(\X \setminus
%     \X_\rho)} = \Sub{\MGU}{(\X \setminus \X_\rho)}$, because if $\Sub{\MGU}{(\X
%     \setminus \X_\rho)}$ were more general than $\Sub{\Unifier}{(\X \setminus
%     \X_\rho)}$ then $\Sub{\MGU}{(\X \setminus \X_\rho)} \cup
%     \Sub{\Unifier}{\X_\rho}$ would be a more general unifier of $\Prob'$ than
%     $\sigma$. So we will write
%     \begin{align*}
%         \Unifier &= \Unifier[0] \uplus \Sub{\Unifier}{\X_\rho} \\
%         \MGU &= \Unifier[0] \uplus \Sub{\MGU}{\X_\rho}
%     \end{align*}
%
%     \todo[inline]{Finish this proof!}
% \end{proof}

At this point, we would like to go on to prove the converse, i.e., that if
$\Prob \InfTo{\INF\BC} \Prob'$ and $\Unifier$ is a most general unifier of
$\Prob'$, then $\Unifier$ is a most general unifier of $\Prob$. Unfortunately,
this is not the case. Consider the following counterexample.

\begin{Example}
    We define a unification problem $\Prob$ as follows:
    \begin{align*}
        \Prob := \{ \,
        & \Vl{u} \ueq \Bc(\Ve{v}, \Vl{w}), \;
        \Vl{u} \ueq \Bc(\Ve{x}, \Vl{y}), \;
        \Vl{u} \ueq \Cons(\Ve{u}', \Vl{u}''), \\
        & \Vl{w} \ueq \Cons(\Ve{w}', \Vl{z}), \;
        \Vl{y} \ueq \Cons(\Ve{y}', \Vl{z}), \;
        \Vl{u}'' \ueq \Bc(\Ve{u}', \Vl{z}), \\
        & \Ve{u}' \ueq \Ff(\Ve{w}', \Ve{v}), \;
        \Ve{u}' \ueq \Ff(\Ve{y}', \Ve{x}) \, \}
    \end{align*}
    If we apply rule \InfRef{cancel-bc-nondet}, we obtain the following
    problem $\Prob[1]$.
    \begin{align*}
        \Prob[1] := \{ \,
        & \Vl{u} \ueq \Bc(\Ve{v}, \Vl{w}), \;
        \Vl{u} \ueq \Cons(\Ve{u}', \Vl{u}''), \;
        \Vl{w} \ueq \Cons(\Ve{w}', \Vl{z}), \\
        & \Vl{y} \ueq \Cons(\Ve{y}', \Vl{z}), \;
        \Vl{u}'' \ueq \Bc(\Ve{u}', \Vl{z}), \;
        \Ve{u}' \ueq \Ff(\Ve{w}', \Ve{v}), \\
        & \Ve{u}' \ueq \Ff(\Ve{y}', \Ve{x}), \;
        \Ve{x} \ueq \Ve{v}, \;
        \Vl{y} \ueq \Vl{w} \, \}
    \end{align*}
    We now apply deterministic rules to transform $\Prob[1]$ into dag-solved form.
    \begin{align*}
        \Prob[1] \InfTo[*]{\text{\InfRef{var-elim}}} \{ \,
        & \Vl{u} \ueq \Bc(\Ve{v}, \Vl{w}), \;
        \Vl{u} \ueq \Cons(\Ve{u}', \Vl{u}''), \;
        \Vl{w} \ueq \Cons(\Ve{w}', \Vl{z}), \\
        & \Vl{w} \ueq \Cons(\Ve{y}', \Vl{z}), \;
        \Vl{u}'' \ueq \Bc(\Ve{u}', \Vl{z}), \;
        \Ve{u}' \ueq \Ff(\Ve{w}', \Ve{v}), \\
        & \Ve{u}' \ueq \Ff(\Ve{y}', \Ve{v}), \;
        \Ve{x} \ueq \Ve{v}, \;
        \Vl{y} \ueq \Vl{w} \, \}
        \displaybreak[0] \\[1ex]
        %
        \hphantom{\Prob[1]} \InfTo{\text{\InfRef{cancel-cons}}} \{ \,
        & \Vl{u} \ueq \Bc(\Ve{v}, \Vl{w}), \;
        \Vl{u} \ueq \Cons(\Ve{u}', \Vl{u}''), \;
        \Vl{w} \ueq \Cons(\Ve{w}', \Vl{z}), \\
        & \Vl{u}'' \ueq \Bc(\Ve{u}', \Vl{z}), \;
        \Ve{u}' \ueq \Ff(\Ve{w}', \Ve{v}), \;
        \Ve{u}' \ueq \Ff(\Ve{y}', \Ve{v}), \\
        & \Ve{x} \ueq \Ve{v}, \;
        \Vl{y} \ueq \Vl{w}, \;
        \Ve{y}' \ueq \Ve{w}' \, \}
        \displaybreak[0] \\[1ex]
        %
        \hphantom{\Prob[1]} \InfTo{\text{\InfRef{var-elim}}} \{ \,
        & \Vl{u} \ueq \Bc(\Ve{v}, \Vl{w}), \;
        \Vl{u} \ueq \Cons(\Ve{u}', \Vl{u}''), \;
        \Vl{w} \ueq \Cons(\Ve{w}', \Vl{z}), \\
        & \Vl{u}'' \ueq \Bc(\Ve{u}', \Vl{z}), \;
        \Ve{u}' \ueq \Ff(\Ve{w}', \Ve{v}), \;
        \Ve{x} \ueq \Ve{v}, \;
        \Vl{y} \ueq \Vl{w}, \\
        & \Ve{y}' \ueq \Ve{w}' \, \}
        \displaybreak[0] \\[1ex]
        %
        \hphantom{\Prob[1]} \InfTo{\text{\InfRef{splitting}}} \{ \,
        & \Vl{u} \ueq \Cons(\Ve{u}', \Vl{u}''), \;
        \Vl{w} \ueq \Cons(\Ve{w}', \Vl{z}), \;
        \Vl{u}'' \ueq \Bc(\Ve{u}', \Vl{z}), \\
        & \Ve{u}' \ueq \Ff(\Ve{w}', \Ve{v}), \;
        \Ve{x} \ueq \Ve{v}, \;
        \Vl{y} \ueq \Vl{w}, \;
        \Ve{y}' \ueq \Ve{w}' \, \}
    \end{align*}
    The resulting problem, which we shall call $\hat{\Prob}_1$, has the
    following most general unifier.
    \begin{align*}
        \Mgu(\hat{\Prob}_1) := \{ \,
        & \Vl{u} \mapsto \Cons(\Ff(\Ve{w}', \Ve{v}), \Bc(\Ff(\Ve{w}', \Ve{v}), \Vl{z})), \\
        & \Vl{u}'' \mapsto \Bc(\Ff(\Ve{w}', \Ve{v}), \Vl{z}), \;
        \Vl{w} \mapsto \Cons(\Ve{w}', \Vl{z}), \\
        & \Vl{y} \mapsto \Cons(\Ve{w}', \Vl{z}), \;
        \Ve{u} \mapsto \Ff(\Ve{w}', \Ve{v}), \\
        & \Ve{x} \mapsto \Ve{v}, \;
        \Ve{y}' \mapsto \Ve{w}' \, \}
    \end{align*}

    However, if we had applied the deterministic rule \InfRef{push-bc-below} to
    $\Prob$ from the start, we would have instead obtained the following problem
    $\Prob[2]$.
    \begin{align*}
        \Prob[2] := \{ \,
        & \Vl{u} \ueq \Cons(\Ve{u}', \Vl{u}''), \;
        \Vl{w} \ueq \Cons(\Ve{w}', \Vl{z}), \;
        \Vl{y} \ueq \Cons(\Ve{y}', \Vl{z}), \\
        & \Vl{u}'' \ueq \Bc(\Ve{u}', \Vl{z}), \;
        \Ve{u}' \ueq \Ff(\Ve{w}', \Ve{v}), \;
        \Ve{u}' \ueq \Ff(\Ve{y}', \Ve{x}) \, \}
    \end{align*}

    \todo[inline]{Where do we go from here?}
\end{Example}


\section{Termination of $\INF\BC$}

To prove termination of $\INF\BC$, we first define two new relations on list
variables.

\begin{Definition}
    Let $\Prob$ be a $(\BCUF)$-unification problem, and let $\Vl{u}$ and
    $\Vl{v}$ be variables in $\Var(\Prob)$. We define $\Sym{\EqLabel}$ to be
    the smallest equivalence relation on the list variables of $\Prob$ such
    that $\Vl{u} \Sym{\EqLabel} \Vl{v}$ if either $(\Vl{u} \ueq \Vl{v}) \in
    \Prob$ or $\{\Vl{u} \ueq \Nil, \, \Vl{v} \ueq \Nil\} \subseteq \Prob$.
\end{Definition}

Variables are related by $\Sym{\EqLabel}$ are equivalent in the sense that they
must be assigned equal values in any unifier.

\begin{Definition}
    Let $\Prob$ be a $(\BCUF)$-unification problem. We define $\Sym{\beta}$ to
    be the smallest equivalence relation on the list variables of $\Prob$ such
    that:
    \begin{enumerate}[(i)]
        \item ${\Sym{\Bc}} \cup {\Sym{\EqLabel}} \; \subseteq \; {\Sym{\beta}}$
        \item For $\Vl{u}$, $\Vl{v}$, $\Vl{x}$, and $\Vl{y}$ in $\Var(\Prob)$,
            if $\Vl{u} \Gt{\Cons} \Vl{x}$ and $\Vl{v} \Gt{\Cons} \Vl{y}$, then
            $\Vl{u} \Sym{\beta} \Vl{v}$ if and only if $\Vl{x} \Sym{\beta}
            \Vl{y}$.
    \end{enumerate}

    We write $\Var(\Prob)/{\Sym{\beta}}$ for the set of
    $\Sym{\beta}$-equivalence classes of $\Var(\Prob)$, and we write
    $\BetaEqC{\Vl{u}}$ for the equivalence class containing the variable
    $\Vl{u}$.
\end{Definition}

The $\Sym{\beta}$-equivalence classes of $\Prob$ correspond to ``levels'' in
the dependency graph, with $\Bc$-labelled edges running between nodes at the
same level on $\Cons$ ``chains''. We will now make these ideas more formal.

\begin{Definition}
    Let $\Prob$ be a $(\BCUF)$-unification problem. We extend the definition of
    the $\Gt{\Cons}$ relation to $\Sym{\beta}$-equivalence classes so that
    $\EqClass{\Vl{u}}{\beta} \Gt{\Cons} \EqClass{\Vl{v}}{\beta}$ if and only if
    there are variables $\Vl{u}$ and $\Vl{v}$ in $\EqClass{\Vl{u}}{\beta}$ and
    $\EqClass{\Vl{v}}{\beta}$, respectively, such that $\Vl{u} \Gt{\Cons}
    \Vl{v}$.
\end{Definition}

\begin{Lemma}\label{lemma:equiv-class-bound}
    Let $\Prob$ be a $(\BCUF)$-unification problem. For any problem $\Prob'$
    such that $\Prob \InfTo[*]{\INF\BC} \Prob'$,
    \[ 0 \; \leq \;
       \left(|\Var(\Prob') / {\Sym{\beta}}| - |\Var(\Prob) / {\Sym{\beta}}|\right)
       \; \leq \;
       \left(|\Var(\Prob) / {\Sym{\beta}}| \cdot |\Var(\Prob)|\right). \]
\end{Lemma}

\begin{proof}
    The inference rules of $\INF\BC$ preserve $\Sym{\beta}$-equivalence classes,
    in the sense that if $\Vl{u}$ and $\Vl{v}$ are variables in $\Prob$, then
    $\Vl{u} \Sym{\beta} \Vl{v}$ in $\Prob$ if and only if $\Vl{u} \Sym{\beta}
    \Vl{v}$ in $\Prob'$. Therefore, no equivalence classes are ``lost''; only
    new equivalence classes are added.
    The only rules in $\INF\BC$ that increase the number of $\Cons$ edges, and
    thus potentially the number of $\Sym{\beta}$-equivalence classes, are
    \InfRef{push-bc-below}, \InfRef{splitting}, and \InfRef{non-nil-nondet}.

    In the first case, since $\Vl{u}$ is in $\Nonnil$, $\Vl{u}$ and $\Vl{u}''$
    must be elements of existing $\Sym{\beta}$-equivalence classes. Since
    $\Vl{u}''$ and $\Vl{z}$ share an equivalence class, so do $\Vl{u}$,
    $\Vl{w}$, and $\Vl{y}$. Thus the number of equivalence classes does not
    increase.

    In the second case, since $\Vl{w}$ and $\Vl{y}''$ are in the same
    $\Sym{\beta}$-equivalence class, so are $\Vl{u}$ and $\Vl{y}$. Thus the
    number of equivalence classes does not increase.

    The third case is almost the same as the first, except it is not required
    that $\Vl{u}$ be in $\Nonnil$. Thus the variable $\Vl{u}''$ may be in a new
    $\Sym{\beta}$-equivalence class. However, since only two variables are
    added to the new equivalence class, $|\BetaEqC{\Vl{u}}| >
    |\BetaEqC{\Vl{u}''}|$. Thus the number of new equivalence classes that can
    be created below $\BetaEqC{\Vl{u}}$ is bounded by $|\Var(\Prob)|$, and the
    total number of new equivalence classes is bounded by $|\Var(\Prob) /
    {\Sym{\beta}}| \cdot |\Var(\Prob)|$.
\end{proof}

\begin{Lemma}\label{lemma:gt-cons-partial-order}
    Let $\Prob$ be a $(\BCUF)$-unification problem such that none of the
    failure rules of $\INF\BC$ is applicable. Then $\Gt[+]{\Cons}$ is a
    wel{}l-founded strict partial order on the $\Sym{\beta}$-equivalence
    classes of $\Var(\Prob)$.
\end{Lemma}

\begin{proof}
    The relation $\Gt[+]{\Cons}$ is by its definition transitive. Since failure
    rule \InfRef{occurs-check} does not apply, $\Gt[+]{\Cons}$ is also
    irreflexive. Thus it is asymmetric and we need only show well-foundedness.

    By \cref{lemma:equiv-class-bound} the number of $\Sym{\beta}$-equivalence
    classes is finite. Suppose $\Gt[+]{\Cons}$ were not well-founded. There
    would have to be an equivalence class $\EqClass{\Vl{u}}{\beta}$ such that
    $\EqClass{\Vl{u}}{\beta} \Gt[+]{\Cons} \EqClass{\Vl{u}}{\beta}$. But then
    there would be a variable $\Vl{u} \in \EqClass{\Vl{u}}{\beta}$ such that
    $\Vl{u} \Gt{\Len} \Vl{u}$ and \InfRef{occurs-check} would apply, which is
    a contradiction. Thus $\Gt[+]{\Cons}$ is well-founded on
    $\Sym{\beta}$-equivalence classes.
\end{proof}

\begin{Lemma}\label{lemma:total-connected-comp}
    Let $\Prob$ be a $(\BCUF)$-unification problem such that $\Gt[+]{\Cons}$ is
    a strict partial order on the $\Sym{\beta}$-equivalence classes of
    $\Var(\Prob)$. Let $\DepGraph(\Prob)$ be the dependency graph of $\Prob$,
    and let $C$ be a connected component of $\DepGraph(\Prob)$. Then
    $\Gt[+]{\Cons}$ is total on the $\Sym{\beta}$-equivalence classes of nodes
    in $C$.
\end{Lemma}

\begin{proof}
    Consider two $\Gt[+]{\Cons}$-incomparable $\Sym{\beta}$-equivalence classes
    $\BetaEqC{\Vl{u}}$ and $\BetaEqC{\Vl{v}}$ of nodes in $C$. Since these
    equivalence classes are in the same connected component, there must be a
    node $\Vl{w}$ such that there is a path to or from $\Vl{w}$ to some
    $\Vl{u}$ in $\BetaEqC{\Vl{u}}$ and some $\Vl{v}$ in $\BetaEqC{\Vl{v}}$. If
    there is a path from $\Vl{u}$ to $\Vl{w}$ to $\Vl{v}$ or from $\Vl{v}$ to
    $\Vl{w}$ to $\Vl{u}$, then $\EqClass{\Vl{u}}{\beta} \Gt[+]{\Cons}
    \EqClass{\Vl{v}}{\beta}$ or $\EqClass{\Vl{v}}{\beta} \Gt[+]{\Cons}
    \EqClass{\Vl{u}}{\beta}$, respectively. So, if $\BetaEqC{\Vl{u}}$ and
    $\BetaEqC{\Vl{v}}$ are incomparable, then either $\BetaEqC{\Vl{u}}
    \Gt[+]{\Cons} \BetaEqC{\Vl{w}} \Lt[+]{\Cons} \BetaEqC{\Vl{v}}$ or
    $\BetaEqC{\Vl{u}} \Lt[+]{\Cons} \BetaEqC{\Vl{w}} \Gt[+]{\Cons}
    \BetaEqC{\Vl{v}}$.

    In the first case when $\BetaEqC{\Vl{u}} \Gt[+]{\Cons} \BetaEqC{\Vl{w}}
    \Lt[+]{\Cons} \BetaEqC{\Vl{v}}$, there must be variables $\Vl{w}_1$ and
    $\Vl{w}_2$ in $\BetaEqC{\Vl{w}}$ such that $\Vl{u} \Gt[+]{\Cons} \Vl{w}_1
    \Sym{\beta} \Vl{w}_2 \Lt[+]{\Cons} \Vl{v}$. But then $\Vl{u} \Sym{\beta}
    \Vl{v}$ and $\BetaEqC{\Vl{u}} = \BetaEqC{\Vl{v}}$. Similarly, in the second
    case when $\BetaEqC{\Vl{u}} \Gt[+]{\Cons} \BetaEqC{\Vl{w}} \Lt[+]{\Cons}
    \BetaEqC{\Vl{v}}$, there must be variables $\Vl{w}_1$ and $\Vl{w}_2$ in
    $\BetaEqC{\Vl{w}}$ such that $\Vl{u} \Lt[+]{\Cons} \Vl{w}_1 \Sym{\beta}
    \Vl{w}_2 \Gt[+]{\Cons} \Vl{v}$. But again, then $\Vl{u} \Sym{\beta} \Vl{v}$
    and $\BetaEqC{\Vl{u}} = \BetaEqC{\Vl{v}}$.

    Thus, if $\BetaEqC{\Vl{u}}$ and $\BetaEqC{\Vl{v}}$ are incomparable in
    $\Gt[+]{\Cons}$, then $\BetaEqC{\Vl{u}} = \BetaEqC{\Vl{v}}$. Therefore,
    $\Gt[+]{\Cons}$ is a strict total order on $\Sym{\beta}$-equivalence
    classes of nodes in $C$. \end{proof}

\begin{Definition}
    Let $\Prob$ be a $(\BCUF)$-unification problem such that $\Gt[+]{\Cons}$ is
    a well-founded strict partial order on the $\Sym{\beta}$-equivalence
    classes of $\Var(\Prob)$. Let $\EqClass{\Vl{u}}{\beta}$ be a
    $\Sym{\beta}$-equivalence class. We define the $\Cons$-depth of
    $\EqClass{\Vl{u}}{\beta}$, written $\ConsDepth(\EqClass{\Vl{u}}{\beta})$, as
    follows:
    \begin{enumerate}[(i)]
        \item If $\EqClass{\Vl{u}}{\beta}$ is maximal in $\Gt[+]{\Cons}$, then
            $\ConsDepth(\EqClass{\Vl{u}}{\beta}) := 0$.
        \item Otherwise, $\ConsDepth(\EqClass{\Vl{u}}{\beta}) :=
            \ConsDepth(\EqClass{\Vl{v}}{\beta}) + 1$, where $\EqClass{\Vl{v}}{\beta}
            \Gt{\Cons} \EqClass{\Vl{u}}{\beta}$.
    \end{enumerate}
    Note that, in the second case, $\EqClass{\Vl{v}}{\beta}$ must be unique by
    \cref{lemma:total-connected-comp}.

    For convenience, we extend this definition to list variables in the obvious
    way, so that $\ConsDepth(\Vl{u}) := \ConsDepth(\EqClass{\Vl{u}}{\beta})$.
\end{Definition}

\begin{Theorem}[Termination of $\INF\BC$]\label{theorem:inf-bc-terminates}
    The relation $\InfTo{\INF\BC}$ is well-founded.
\end{Theorem}

\begin{proof}
    Suppose $\InfTo{\INF\BC}$ is not well-founded. Then there must be an
    infinitely descending chain
    \[ \Prob[0] \InfTo{\INF\BC} \Prob[1] \InfTo{\INF\BC} \dotsb \InfTo{\INF\BC}
    \Prob[i] \InfTo{\INF\BC} \dotsb \]
    Since there are a finite number of inference rules, at least one inference
    rule from $\INF\BC$ must be applied infinitely many times. Clearly this
    rule must not be a failure rule, since those rules cause immediate
    termination.

    For the remaining rules, consider the measure function $\phi\colon
    \ProbType{\BCUF} \to \Nat^4$ from $(\BCUF)$-unification problems to
    quadruples of natural numbers given by:
    \begin{align*}
        \phi(\Prob[i]) &:= \left(
            \FreeClasses, \;
            \FreeDepth, \;
            |\DepGraph(\Prob[i])|, \;
            |\Prob[i]|
        \right)
        \intertext{where}
        \FreeClasses &:= \MaxClasses - |\Var(\Prob[i])/{\Sym{\beta}}| \\[4pt]
        \FreeDepth &:= \sum \, \{
            \MaxClasses - \ConsDepth(\Vl{u}) \mid
            (\Vl{u} \ueq \Bc(\Ve{v}, \Vl{w})) \in \Prob[i]
        \} \\[4pt]
        \MaxClasses &:= |\Var(\Prob[0])/{\Sym{\beta}}| \cdot
            (1 + |\Var(\Prob[0])|)
    \end{align*}

    The first component, $\FreeClasses$, measures the number of number of
    unused $\Sym{\beta}$-equivalence class ``slots''. The second component,
    $\FreeDepth$, is intuitively a measure of how many times a $\Bc$-edge can
    be ``lowered'' by rules like \InfRef{splitting}. Note that $\MaxClasses$,
    which measures the maximum number of $\Sym{\beta}$-equivalence classes, is
    fixed by the initial input probem $\Prob[0]$ thanks to the bound given by
    \cref{lemma:equiv-class-bound}.

    The function $\phi$ is only defined when $\ConsDepth$ is defined --- i.e.,
    only when $\Gt[+]{\Cons}$ is a well-founded strict partial order on the
    $\Sym{\beta}$-equivalence classes. Since the failure rules must not apply,
    this is the case according to \cref{lemma:gt-cons-partial-order}.

    For the remaining inference rules in $\INF\BC$, we will show that if
    $\Prob[i] \InfTo{\INF\BC} \Prob[i+1]$ then $\phi(\Prob[i]) \Gt{\Lex}
    \phi(\Prob[i+1])$, where $\Gt{\Lex}$ is the lexicographic order on $\Nat^4$
    induced by $\Gt{}$ on $\Nat$.

    \begin{itemize}[align=left]
        \item[(\InfRef{triv-elim})] This rule preserves the first three
            components and decreases the fourth.

        \item[(\InfRef{var-elim}--\InfRef{cancel-cons})] These rules preserve
            or decrease the first two components and decrease the third.
    \end{itemize}

    \begin{itemize}[align=left]
        \item[(\InfRef{nil-soln-1}--\InfRef{semi-cancel-bc})] These rules all
            preserve or decrease the first two components and decrease the
            third.

        \item[(\InfRef{push-bc-below}--\InfRef{splitting})] While these rules
            do add several new variables and edges to the dependency graph,
            they add no new $\Sym{\beta}$-equivalence classes, and they remove
            a $\Bc$-equation and add a new one with a variable that has a
            greater $\Cons$-depth. Thus \InfRef{push-bc-below} and
            \InfRef{splitting} preserve the first component and decrease the
            second.

        \item[(\InfRef{nil-soln-nondet})] This rule behaves the same as
            \InfRef{nil-soln-1}--\InfRef{semi-cancel-bc}, preserving or
            decreasing the first two components and decreasing the third.

        \item[(\InfRef{non-nil-nondet})] This rule either behaves the same as
            \InfRef{push-bc-below}, preserving the first component and
            decreasing the second, or it creates a new
            $\Sym{\beta}$-equivalence class and decreases the first component.

        \item[(\InfRef{cancel-bc-nondet})] This rule behaves the same as
            \InfRef{semi-cancel-bc}, preserving or decreasing the first two
            components and decreasing the third.
    \end{itemize}

    Each non-failure inference rule in $\INF\BC$ causes the measure of the
    problem to strictly decrease. Since $\Gt{\Lex}$ is well-founded on $\Nat^4$,
    this cannot continue infinitely, so no rule in $\INF\BC$ can be applied
    infinitely many times. Thus $\InfTo{\INF\BC}$ is well-founded.
\end{proof}



\section{Algorithm $\AlgBC$}\label{sec:bc-algorithm}

Our algorithm is split into three phases. First we use the inference system
$\INF\BC$ to transform the list equations of a $(\BCUF)$-unification problem
$\Prob$ into dag-solved form. We then pass the element equations to an
algorithm for the $\FF$-unification problem. Finally, we combine the list
equations with the element unifier to produce a unifier.

\subsection*{Input and Output}

The input of our algorithm is a $(\BCUF)$-unification problem $\Prob$. The
output is a unifier $\Unifier$ of $\Prob$, or $\Fail$ if no unifier exists.

\subsection*{Phase 1: Solving List Equations}

The first phase of the algorithm is to apply $\INF\BC$ to the input problem
exhaustively. Whenever we have a choice among nondeterministic rules, we
guess a rule to apply as follows:
\begin{enumerate}[(1)]
    \item If one of F1 or F2 is applicable to $\Prob$, pick that rule.
    \item Otherwise, if one of L1--L3 is applicable, pick the lowest
        numbered applicable rule.
    \item Otherwise, if one of BC1--BC6 is applicable, pick the lowest
        numbered applicable rule.
    \item Otherwise, if one of BC7--BC9 is applicable, guess one of
        the rules to apply.
    \item Otherwise terminate.
\end{enumerate}

Thus we reduce $\Prob$ to a problem $\Prob'$ such that $\Prob
\InfTo[!]{\INF\BC} \Prob'$, and by \cref{lemma:infl-dag-solved},
$\ListEq(\Prob')$ is in dag-solved form. Let $\Unifier[\ListEq]$ be the most
general unifier of $\ListEq(\Prob')$ as given in \cref{sec:std-form}. At this
point, we can consider the list equations ``solved'' and shift our attention to
the element equations.

If we reach $\Fail$, we return $\Fail$ and stop. Note that our algorithm is
nondeterministic, so one particular path may fail while another will
successfully find a unifier. The algorithm can be made deterministic by
implementing a backtracking search procedure.

\subsection*{Phase 2: Solving Element Equations}

In this phase of the algorithm we solve $\EltEq(\Prob')$. We assume that we
have an algorithm $\AlgF$ for the $\FF$-unification problem. Recall that we can
think of $\Alg{\FF}$ as a function from unification problems to sets of
substitutions.  Since $\FF$ is an element theory, $\FF$-unification is
finitary, so $\AlgF(\EltEq(\Prob'))$ will be finite. We nondeterministically
guess a unifier $\Unifier[\EltEq]$ from $\AlgF(\EltEq(\Prob'))$.

If $\AlgF(\EltEq(\Prob'))$ is empty, we return $\Fail$ and stop. Again, a
different path through the algorithm could yield a unifier.

\subsection*{Phase 3: Combining Results}

Now we have our unifiers $\Unifier[\ListEq]$ and $\Unifier[\EltEq]$ of
$\ListEq(\Prob')$ and $\EltEq(\Prob')$, respectively. We combine them into a
single unifier $\Unifier := \Unifier[\EltEq] \Compose \Unifier[\ListEq]$. We
then return this unifier and stop.



\section{Soundness and Completeness of $\Alg{\BC}$}\label{sec:bc-sound-complete}

\begin{Theorem}[Soundness of $\AlgBC$]
    Let $\Prob$ be a $(\BCUF)$-unification problem, and let $\Unifier$ be a
    substitution such that $\AlgBC(\Prob, \Unifier)$. Then $\Unifier$ is a
    $(\BCUF)$-unifier of $\Prob$.
\end{Theorem}

\begin{proof}
    If $\AlgBC(\Prob, \Unifier)$, then $\Unifier = \Unifier[\EltEq] \Compose
    \Unifier[\ListEq]$, where $\Prob \InfTo[!]{\INF\BC} \Prob'$, and
    $\Unifier[\ListEq]$ and $\Unifier[\EltEq]$ are unifiers of
    $\ListEq(\Prob')$ and $\EltEq(\Prob')$, respectively. Since $\Unifier$ is
    an instance of $\Unifier[\ListEq]$, it is also a unifier of
    $\ListEq(\Prob')$. Since its domain contains only list variables,
    $\Unifier[\ListEq]$ is equivalent to $\Ident$ when restricted to element
    variables. Thus $\Unifier$ is equivalent to $\Unifier[\EltEq]$ on element
    variables and is a unifier of $\EltEq(\Prob')$. Since $\Prob' =
    \ListEq(\Prob') \uplus \EltEq(\Prob')$, $\Unifier$ is a unifier of
    $\Prob'$, so by the soundness of $\INF\BC$ from \cref{thm:inf-bc-sound},
    $\Unifier$ is a unifier of $\Prob$.
\end{proof}

\begin{Theorem}[Completeness of $\AlgBC$]
    Let $\Prob$ be a $(\BCUF)$-unification problem, and let $\Unifier$ be a
    unifier of $\Prob$. Then $\AlgBC(\Prob, \Unifier')$ such that $\Unifier
    \LessGeneralThan \Unifier'$.
\end{Theorem}

\begin{proof}
    If $\Unifier$ is a unifier of $\Prob$, then by the completeness of
    $\INF\BC$ from \cref{thm:inf-bc-complete}, there is a problem $\Prob'$ such
    that $\Prob \InfTo[!]{\INF\BC} \Prob'$ and $\Unifier$ is a unifier of
    $\Prob'$. Our algorithm would find the most general unifier
    $\Unifier[\ListEq]$ of $\ListEq(\Prob')$, and $\Unifier \LessGeneralThan
    \Unifier[\ListEq]$. We would then guess a unifier $\Unifier[\EltEq]$ from
    $\AlgF(\EltEq(\Prob'))$.
\end{proof}


\section{Termination of $\Alg{\BC}$}\label{sec:bc-termination}

The termination of our algorithm follows directly from the termination of
$\INF\BC$.

\section{Runtime Analysis of $\Alg{\BC}$}\label{sec:bc-runtime-analysis}

\begin{Theorem}
    Let $\Prob$ and $\Prob'$ be $(\BCUF)$-unification problems such that
    $|\Prob| = n$ and $\Prob \InfTo[!]{\INF\BC} \Prob'$. Then $\Prob
    \InfTo[k]{\INF\BC} \Prob'$ such that $k < n^4$.
\end{Theorem}

\begin{proof}
    In the proof of \cref{theorem:inf-bc-terminates} we defined a measure
    function $\phi\colon \ProbType{\BCUF} \to \Nat^4$ from
    $(\BCUF)$-unification problems to quadruples of natural numbers given by:
    \begin{align*}
        \phi(\Prob[i]) &:= \left(
            \FreeClasses, \;
            \FreeDepth, \;
            |\DepGraph(\Prob[i])|, \;
            |\Prob[i]|
        \right)
        \intertext{where}
        \FreeClasses &:= \MaxClasses - |\Var(\Prob[i])/{\Sym{\beta}}| \\[4pt]
        \FreeDepth &:= \sum \, \{
            \MaxClasses - \ConsDepth(\Vl{u}) \mid
            (\Vl{u} \ueq \Bc(\Ve{v}, \Vl{w})) \in \Prob[i]
        \} \\[4pt]
        \MaxClasses &:= |\Var(\Prob[0])/{\Sym{\beta}}| \cdot
            (1 + |\Var(\Prob[0])|)
    \end{align*}
    In that proof, we showed that if $\Prob[i] \InfTo{\INF\BC} \Prob[i+1]$ then
    $\phi(\Prob[i]) \Gt{\Lex} \phi(\Prob[i+1])$, where $\Gt{\Lex}$ is the
    lexicographic order on $\Nat^4$ induced by $\Gt{}$ on $\Nat$.
\end{proof}

\section{Dependency Graph Transformations}\label{sec:dep-graph-trans}

